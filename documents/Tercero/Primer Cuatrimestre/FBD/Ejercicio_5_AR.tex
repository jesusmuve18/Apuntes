\documentclass[12pt]{article}

\input{../../../_assets/preambulo.tex}
\usepackage{xkeyval} % Para el paso de argumentos

\usepackage{graphicx}

% Definir la carpeta de las imágenes
\graphicspath{{../_assets}{../../_assets}}

% Definir el comando \portada
\makeatletter
\define@key{portada}{titulo}{\def\titulo{#1}}
\define@key{portada}{subtitulo}{\def\subtitulo{#1}}
\define@key{portada}{autor}{\def\autor{#1}}
\define@key{portada}{año}{\def\año{#1}}

\newcommand*{\portada}[1][]{%
    % Definimos las claves y sus valores por defecto
  \setkeys{portada}{%
    titulo=Sin Título,%
    subtitulo=Sin Subtítulo,%
    autor=Autor Desconocido,%
    año=Sin Año, #1}%
    \begin{titlepage}
        \centering
        {\includegraphics[width=0.2\textwidth]{Logo-UGR-Black.png}\par}
        \vspace{1cm}
        {\bfseries\LARGE Universidad de Granada \par}
        \vspace{1cm}
        {\scshape\Large Doble Grado en Ingeniería Informática y Matemáticas \par}
        \vspace{3cm}
        {\scshape\Huge \titulo \par}
        \vspace{3cm}
        {\itshape\Large \subtitulo \par}
        \vfill
        {\Large Autor: \par}
        {\Large \autor \par}
        \vfill
        {\Large \año \par}
    \end{titlepage}%
}
\usepackage{ wasysym }

\begin{document}

\portada[%
        titulo=Fundamentos de Bases de Datos,
        subtitulo=Ejercicio 5 del Seminario 4 (Álgebra Relacional),
        autor=Jesús Muñoz Velasco,
        año=Curso 2024-2025]

\setcounter{ejercicio}{4}

\begin{ejercicio}
    Parejas $<$cargo, marca$>$ entre las que se ha dado alguna reparación.
    \endsquare
    \begin{align*}
        \pi_{cargo,\ marca}(((\text{REPARA } \Bowtie \text{ MECÁNICO})\Bowtie \text{ VEHÍCULO})\Bowtie \text{ MODELO})
    \end{align*}
    Además, se ha comentado durante la realización del ejercicio que el número máximo de tuplas que habrá en la tabla resultado será el número de tuplas de la tabla ``REPARA''.
\end{ejercicio}
    

\end{document}