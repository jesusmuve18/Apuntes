\section{Trabajo de clase del lunes 28 de Octubre de 2024}

\begin{ejercicio}
    Sea $X=(X_1,X_2)$ y dada la tabla

    \begin{center}
        \begin{tabular}{c | ccc | c}  
            $X_2\backslash X_1$ & -1 & 0 & 1& \\ 
            \hline
            -2 & \nicefrac{1}{6} & \nicefrac{1}{12} & \nicefrac{1}{6}& \\
            1 & \nicefrac{1}{6} & \nicefrac{1}{12} & \nicefrac{1}{6}& \\
            2 & \nicefrac{1}{12} & 0 & \nicefrac{1}{12}& \\
            \hline
            &&&&\\
        \end{tabular}
    \end{center}
    Hallar la función masa de probabilidad conjunta de $Y=(|X_1|, X_2^2)$.\\

    \begin{align*}
        P[Y=(0,1)] &= P[X_1=0, X_2=1] = \nicefrac{1}{12}\\
        P[Y=(1,1)] &= P[X_1=-1, X_2 = 2] + P[X_1=1, X_2=1] = \nicefrac{1}{6} + \nicefrac{1}{6} = \nicefrac{1}{3}\\
        P[Y=(0,4)] &= P[X_1=1, X_2=-2] + P[X_1=0, X_2=2] = \nicefrac{1}{12} + 0 = \nicefrac{1}{12}\\
        P[Y=(1,4)] &= P[X_1=-1, X_2=-2] + P[X_1=1, X_2=-2]\\
        &\phantom{=} +P[X_1=1, X_2=2] + P[X_1=-1, X_2=2] = \nicefrac{1}{6} + \nicefrac{1}{6} + \nicefrac{1}{12} + \nicefrac{1}{12} = \nicefrac{1}{2}
    \end{align*}

    Y obtenemos la siguiente tabla de fmp:
    \begin{center}
        \begin{tabular}{c | cc | c}  
            $X_2^2\backslash |X_1|$ & 0 & 1& \\ 
            \hline
            1 & \nicefrac{1}{12}& \nicefrac{1}{3}\\
            4 & \nicefrac{1}{12}& \nicefrac{1}{2}\\
            \hline
            &&&\\
        \end{tabular}
    \end{center}

\end{ejercicio}

\begin{ejercicio}
    Sea $X=(X_1, X_2)$ el vector aleatorio continuo que tiene por función de densidad
    \begin{gather*}
        f_X(x_1, x_2) = \lambda \mu e^{-\lambda x_1} e^{-\mu x_2} \ \ \text{ con }\  x_1, x_2>0,\ \  \lambda, \mu>0 \text{ constantes }
    \end{gather*}
    Sea la transformación 
    \begin{gather*}
        Y=\left\{
        \begin{array}{ccc}
            0 & \text{ si } & x_1 > x_2\\
            1 & \text{ si } & x_1 < x_2
        \end{array}
        \right.
    \end{gather*}
    Calcula la función masa de probabilidad de $Y$.\\

    Para ello tendremos que calcular la probabilidad de que $Y$ sea igual a 1 y la probabilidad de que $Y$ sea igual a 0, es decir:
    \begin{align*}
        P[Y=0] &= P[X_1 > X_2] = \lambda \mu \int_0^{\infty}\int_{x_2}^{\infty} e^{-\lambda x_1} e^{-\mu x_2} dx_1 dx_2 \\
        &= \lambda \mu \int_0^{\infty} e^{-\mu x_2} \cdot \left[-\frac{e^{-\lambda x_1}}{\lambda}\right]_{x_2}^{\infty} dx_2 = \mu \int_0^{\infty} e^{-\mu x_2} \cdot (-0 + e^{-\lambda x_2}) dx_2 \\
        &= \mu \int_0^{\infty} e^{(-\lambda -\mu)x_2}dx_2 = \mu \left[\frac{e^{(-\lambda -\mu)x_2}}{\lambda + \mu}\right]_0^{\infty} = \mu \left(-0+\frac{1}{\lambda + \mu}\right) = \frac{\mu}{\lambda + \mu}\\
        P[y=1] &= P[X_1 < X_2] = 1 - P[X_1 > X_2]  =1 - \frac{\mu}{\lambda + \mu}= \frac{\lambda}{\lambda + \mu}
     \end{align*}
\end{ejercicio}

\begin{ejercicio}(Ejercicio 3 de la relación 2 parte 2)

    Tenemos que $f_X(x,y)=1$ para $0<x<1$, $0<y<1$.
    \begin{gather*}
        (Z,T)=(X+Y, X-Y)
    \end{gather*}

    \begin{align*}
        Z=X+Y\\
        \underline{T=X-Y}\\
        Z-T = 2Y
    \end{align*}
    \begin{gather*}
        X=Z-Y = Z - \frac{Z-T}{2} = \frac{Z+T}{2}\\
        Y=\frac{Z-T}{2}\\
    \end{gather*}
    \begin{gather*}
        J_{g^{-1}} = 
        \begin{vmatrix}
            \nicefrac{1}{2} & \nicefrac{1}{2}\\
            \nicefrac{1}{2} & \nicefrac{1}{2}
        \end{vmatrix} = -\nicefrac{1}{2} \neq 0
    \end{gather*}
    Por lo que $g^{-1}(z,t)=\left(\frac{Z+T}{2}, \frac{Z-T}{2}\right) \Rightarrow f_{Z,T}(z,t) = f_{X,Y}\left(\frac{z+t}{2}, \frac{z-t}{2} \right)\cdot |J| = 1 \cdot |-\nicefrac{1}{2}| = \nicefrac{1}{2}$. Por tanto llegamos a 

    \begin{gather*}
        f_{Z,T}(z,t)=\left\{
        \begin{array}{ccc}
            \nicefrac{1}{2}& \text{ si } & 0<z+t<2 \ \vee \  0<z-t<2\\
            0 & \text{ sino } & 
        \end{array}
        \right.
    \end{gather*}
    Pasamos a calcular las marginales:
    \begin{gather*}
        f_Z(z) =\left\{
        \begin{array}{ccc}
            \int_{-z}^{+z} \nicefrac{1}{2}dt=z & \text{ si } & 0<z<1\\
            \int_{z-2}^{2-z}\nicefrac{1}{2}dt = 2-z & \text{ si } & 1<z<2\\
        \end{array}
        \right.
    \end{gather*}
    \begin{gather*}
        f_T(t) =\left\{
        \begin{array}{ccc}
            t+1 & \text{ si } & -1<t<0\\
            1-t & \text{ si } & 0<t<1\\
        \end{array}
        \right.
    \end{gather*}
    Veamos ahora lo que sucede con la siguiente transformación:
    \begin{gather*}
        (Z,T) = \left(\frac{X}{Y}, XY\right)
    \end{gather*}
\end{ejercicio}