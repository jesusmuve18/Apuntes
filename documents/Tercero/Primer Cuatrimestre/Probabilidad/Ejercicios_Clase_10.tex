\section{Trabajo de clase del martes 12 de Noviembre de 2024}

\begin{ejercicio}
    (Continuación Ejercicio 1.9.1) 
    \begin{align*}
        Var[X]=Var[E[X/Y]]+E[Var[X/Y]]
    \end{align*}
    Del ejercicio teníamos la siguiente tabla: 
    \begin{center}
        \begin{tabular}{c | c c | c}
            $X\backslash Y$ & 0 & 1 & $P[X=x_i]$\\
            \hline
            0 & $\nicefrac{3}{10}$ & $\nicefrac{3}{10}$ & $\nicefrac{3}{5}$\\
            1 & $\nicefrac{3}{10}$ & $ \nicefrac{1}{10}$ & $\nicefrac{2}{5}$\\
            \hline
            $P[Y=y_i]$ & $\nicefrac{3}{5}$ & $\nicefrac{2}{5}$ & \\
        \end{tabular}
    \end{center}
    Veamos en primer lugar cuál es la varianza de $X$:
    \begin{align*}
        E[X] &= 0 \cdot \nicefrac{3}{5} + 1 \cdot \nicefrac{2}{5} = \nicefrac{2}{5}\\
        E[X^2] &= 0^2 \cdot \nicefrac{3}{5} + 1^2 \cdot \nicefrac{2}{5} = \nicefrac{2}{5}\\
        Var(X) &= E[X^2]-E[X]^2 = \nicefrac{2}{5} - \left(\nicefrac{2}{5}\right) = \nicefrac{6}{25}
    \end{align*}
    Calculemos lo que nos queda:
    \begin{align*}
        Var(E[X/Y]) &= E[E[X/Y]^2] - (E[E[X/Y]])^2 =\\
        &= E[E[X/Y]^2] - E[X]^2\\
    \end{align*}
    Del ejercicio nos quedaba lo siguiente:
    \begin{gather*}
        E[X/Y] = \left\{
        \begin{array}{ccc}
            \nicefrac{1}{2} & \text{ si } & Y=0\\
            \nicefrac{1}{4} & \text{ si } & Y=1
        \end{array}
        \right.
    \end{gather*}
    Tenemos entonces lo siguiente:
    \begin{align*}
        E[E[X/Y]^2]=\left(\frac{1}{2}\right)^2 \cdot P[Y=0] + \left(\frac{1}{4}\right)^2 \cdot P[Y=1] = \frac{1}{4} \cdot \frac{3}{5} + \frac{1}{16} \cdot \frac{2}{5} = \frac{7}{40} 
    \end{align*}
    \begin{align*}
        E[Var[X/Y]]=E[E[X^2/Y] - E[X/Y]^2] = E[E[X^2/Y]] - E[E[X/Y]^2] = \frac{2}{5} - \frac{7}{40} = \frac{9}{40}
    \end{align*}
\end{ejercicio}

\begin{ejercicio}
    Se lanzan 2 monedas numeradas con las caras 1 y 2. Notemos $X$ a la suma e $Y$ al máximo de las caras obtenidas. Obtener las funciones de regresión mínimo-cuadráticas de $Y$ dado $X$ y de $X$ dado $Y$ así como los errores cuadráticos medios asociados.

    Tenemos la siguiente tabla:
    \begin{center}
        \begin{tabular}{c|cc|c}
            $X\backslash Y$ & 1 & 2 & $P[X=x_i]$\\
            \hline
            2 & $\nicefrac{1}{4}$ & 0 & $\nicefrac{1}{4}$\\
            3 & 0 & $\nicefrac{1}{2}$ & $\nicefrac{1}{2}$\\
            4 & 0 & $\nicefrac{1}{4}$ & $\nicefrac{1}{4}$\\
            \hline
            $P[Y=y_i]$ & $\nicefrac{1}{4}$ & $\nicefrac{3}{4}$\\
        \end{tabular}
    \end{center}
\end{ejercicio}