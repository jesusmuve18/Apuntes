\chapter{Tema 1: Números Complejos}

Existen ecuaciones lineales que no cuentan con solución real, como por ejemplo la conocida $x^2+1=0$. La idea es extender el conjunto de los números reales para resolver todas las ecuaciones polinómicas. Esto fundamenta el Teorema Fundamental del Álgebra (toda ecuación lineal de grado mayor que 0 tiene al menos una raíz).

\section{El cuerpo $\bb{C}$}

Si definimos 
\begin{align*}
    \bb{R}^2= \{(x,y): x,y \in \bb{R}\}
\end{align*}
podemos considerar las siguientes operaciones, para definir un cuerpo:
\begin{itemize}
    \item Suma: $(x,y) + (u,v) = (x+u, y+v)\ \ \forall x,y,u,v\in \bb{R}$.
    \item Producto: $(x,y)(u,v) = (xu - yv, xv + yu) \ \ \forall x,y,u,v \in \bb{R}$
\end{itemize}

Con estas operaciones definidas tenemos que $\bb{R}^2$ con la suma es ub grupo abeliano. El producto es asociativo, conmutativo y distributivo respecto a la suma. 
Además tenemos elementos neutros para la suma y el producto.

Con esto tenemos un cuerpo conmutativo $\bb{C}$. Como conjuntos tenemos que $\bb{C}=\bb{R}^2$.