\documentclass[12pt]{article}

\input{../../../_assets/preambulo.tex}
\usepackage{xkeyval} % Para el paso de argumentos

\usepackage{graphicx}

% Definir la carpeta de las imágenes
\graphicspath{{../_assets}{../../_assets}}

% Definir el comando \portada
\makeatletter
\define@key{portada}{titulo}{\def\titulo{#1}}
\define@key{portada}{subtitulo}{\def\subtitulo{#1}}
\define@key{portada}{autor}{\def\autor{#1}}
\define@key{portada}{año}{\def\año{#1}}

\newcommand*{\portada}[1][]{%
    % Definimos las claves y sus valores por defecto
  \setkeys{portada}{%
    titulo=Sin Título,%
    subtitulo=Sin Subtítulo,%
    autor=Autor Desconocido,%
    año=Sin Año, #1}%
    \begin{titlepage}
        \centering
        {\includegraphics[width=0.2\textwidth]{Logo-UGR-Black.png}\par}
        \vspace{1cm}
        {\bfseries\LARGE Universidad de Granada \par}
        \vspace{1cm}
        {\scshape\Large Doble Grado en Ingeniería Informática y Matemáticas \par}
        \vspace{3cm}
        {\scshape\Huge \titulo \par}
        \vspace{3cm}
        {\itshape\Large \subtitulo \par}
        \vfill
        {\Large Autor: \par}
        {\Large \autor \par}
        \vfill
        {\Large \año \par}
    \end{titlepage}%
}

\begin{document}
    \portada[%
        titulo=Variable Compleja I,
        subtitulo=Resumen,
        autor=Jesús Muñoz Velasco,
        año=Curso 2024-2025]


    % \section{Test de Weierstrass}

    % Sea $\sum\limits_{n\geq 0} f_n$ una serie de funciones complejas definidas en un conjunto $A\subset \bb{C}$, y sea $B\subset A$. Supongamos que, para cada $n\in \bb{N}\cup \{0\}$, existe una constante $M_n\in \bb{R}$ tal que:
    % \begin{gather*}
    %     |f_n(z)| \leq M_n \hspace{1cm} \forall z \in B
    % \end{gather*}
    % Si la serie de números reales $\sum\limits_{n\geq 0} M_n$ es convergente, entonces la serie de funciones $\sum\limits_{n\geq 0} f_n$ converge absoluta y uniformemente en $B$.

    % \section{Teorema de convergencia de Weierstrass}

    % Sea $\Omega$ un abierto no vacío de $\bb{C}$ y para cada $n\in \bb{N}$, sea $f_n\in \cc{H}(\Omega)$. Supongamos que la sucesión $\{f_n\}$ converge uniformemente en cada subconjunto compacto de $\Omega$ a una función $f:\Omega \to \bb{C}$, en particular:
    % \begin{gather*}
    %     f(z) = \lim_{n\to \infty} f_n(z) \hspace{1cm} \forall z \in \Omega
    % \end{gather*}
    % Entonces $f\in \cc{H}(\Omega)$ y para cada $k\in \bb{N}$ se tiene que la sucesión $\{f_n^{(k)}\}$ de las k-ésimas derivadas, converge a la derivada k-ésima $f^{(k)}$, uniformemente en cada subconjunto compacto de $\Omega$, en  particular:
    % \begin{gather*}
    %     f^{(k)}(z) = \lim_{n\to \infty} f_n^{(k)}(z) \hspace{1cm} \forall z \in \Omega,\ \ \forall k \in \bb{N}
    % \end{gather*}

    % \section{Holomorfía de funciones dependientes de un parámetro}

    \section{Tema 3}

    \begin{teo}
        Sea $A$ un subconjunto no vacío de $\bb{C}$ y $f\in \cc{F}(A)$. Como $A\subset \bb{R}^2$, podemos considerar las funciones $u,v:A \to \bb{R}$ definidas, para todo $(x,y)\in A$ por
        \begin{gather*}
            u(x,y) = Re\ f(x+iy) \hspace{1cm} y \hspace{1cm} v(x,y) = Im\ f(x+iy)
        \end{gather*}
        Para $z_0=(x_0,y_0)\in A^\circ$, las siguientes afirmaciones son equivalentes:
        \begin{enumerate}
            \item La función $f$ es derivable en el punto $z_0$.
            \item Las funciones $u$ y $v$ son diferenciables en el punto $(x_0, y_0)$ verificando que 
            \begin{gather*}
                \frac{\partial u}{\partial x} (x_0, y_0) = \frac{\partial v}{\partial y} (x_0, y_0) \hspace{1cm} y \hspace{1cm} \frac{\partial u}{\partial y} (x_0, y_0) = -\frac{\partial v}{\partial x} (x_0, y_0)
            \end{gather*}
            Estas ecuaciones reciben el nombre de \textbf{ecuaciones de Cauchy-Riemann}.
        \end{enumerate}
        Caso de que se cumplan 1. y 2., se tiene:
        \begin{gather*}
            f'(z_0) = \frac{\partial u}{\partial x} (x_0, y_0) + i \frac{\partial v}{\partial x} (x_0, y_0)
        \end{gather*}
    \end{teo}

    \section{Tema 4}

    \begin{teo}[\textbf{Test de Weierstrass}]
        Sea $\sum\limits_{n\geq 0} f_n$ una serie de funciones complejas definidas en un conjunto $A\subset \bb{C}$, y sea $B\subset A$. Supongamos que, para cada $n\in \bb{N}\cup \{0\}$, existe una constante $M_n\in \bb{R}$ tal que:
        \begin{gather*}
            |f_n(z)| \leq M_n \hspace{1cm} \forall z \in B
        \end{gather*}
        Si la serie de números reales $\sum\limits_{n\geq 0} M_n$ es convergente, entonces la serie de funciones $\sum\limits_{n\geq 0} f_n$ converge absoluta y uniformemente en $B$.
    \end{teo}

    \begin{lema}[\textbf{Lema de Abel}] Dado $\rho\in \bb{R}^+$, supongamos que la sucesión $\{|\alpha_n|\rho^n\}$ está mayorada. Entonces la serie de potencias $\sum\limits_{n\geq 0} \alpha_n ( z-a)^n$ converge absolutamente en el disco abierto $D(a, \rho)$ y uniformemente en cada compacto $K$ que esté contenido en dicho disco.
    \end{lema}

    \begin{prop}[\textbf{Fórmula de Cauchy-Hadamard}]
        Sea $R$ el radio de convergencia de la serie $\sum\limits_{n\geq0}\alpha_nz^n$
        \begin{enumerate}
            \item Si la suceción $\{\sqrt[n]{|\alpha_n|}\}$ no está mayorada, entonces $R=0$.
            \item Si $\{\sqrt[n]{|\alpha_n|}\} \to 0$, entonces $R=\infty$.
            \item En otro caso se tiene: $R = \dfrac{1}{\limsup \{\sqrt[n]{|\alpha_n|}\}}$
        \end{enumerate}
    \end{prop}

    \begin{coro}
        Supongamos que $\alpha_n\in \bb{C}^*$ para todo $n\in \bb{N}$ y sea $R$ el radio de convergencia de la serie de potencias $\sum\limits_{n\geq 0} \alpha_n z^n$.
        \begin{enumerate}
            \item Si $\{\alpha_{n+1} / \alpha_n\} \to \infty$, entonces $R=0$.
            \item Si $\{\alpha_{n+1} / \alpha_n\} \to 0$, entonces $R=\infty$.
            \item Si $\{\alpha_{n+1} / \alpha_n\} \to \lambda \in \bb{R}^+$, entonces $R=1/\lambda$.
        \end{enumerate}
    \end{coro}

    \section{Tema 6}

    \begin{teo}[\textbf{Caracterización de la existencia de primitiva}] Sea $\Omega$ un abierto no vacío de $\bb{C}$ y $f\in \cc{C}(\Omega)$. Las siguientes afirmaciones son equivalentes:
    \begin{enumerate}
        \item Existe $F\in \cc{H}(\Omega)$ tal que $F'(z) = f(z)$ para todo $z\in \Omega$.
        \item Para todo camino cerrado $\gamma$ en $\Omega$ se tiene que $\displaystyle\int_\gamma f(z)dz = 0$.
    \end{enumerate}
        
    \end{teo}

    \section{Tema 7}

    \begin{teo}[\textbf{Teorema local de Cauchy}]
        Si $\Omega$ es un dominio estrellado, entonces toda función admite una primitiva en $\Omega$, es decir, existe $F\in \cc{H}(\Omega)$ tal que $F'(z) = f(z)$ para todo $z\in \Omega$. Equivalentemente se tiene
        \begin{gather*}
            \int_\gamma f(z) dz = 0
        \end{gather*}
        Para toda función $f\in \cc{H}(\Omega)$ y todo camino cerrado $\gamma$ en $\Omega$.
    \end{teo}

    \begin{prop}[\textbf{Fórmula de Cauchy}]
        Sean $\Omega$ un abierto de $\bb{C}$ y $f\in \cc{H}(\Omega)$. Dado $a\in \Omega$, sea $r\in \bb{R}^+$ tal que $\overline{D}(a,r)\subset \Omega$. Se tiene entonces:
        \begin{gather*}
            f(z) = \frac{1}{2\pi i } \int_{C(a,r)} \frac{f(w)}{w-z} dw \hspace{1cm} \forall z \in D(a,r)
        \end{gather*}
    \end{prop}

    \section{Tema 8}
    \begin{teo}[\textbf{Desarrollo en serie de Taylor}]
        Si $\Omega$ es un abierto no vacío de $\bb{C}$ y $f\in \cc{H}(\Omega)$, entonces $f$ es analítica en $\Omega$ y, en particular, $f$ es indefinidamente derivable en $\Omega$. Además:
        \begin{enumerate}
            \item Si $\Omega = \bb{C}$, para todo $a\in \bb{C}$, la serie $\sum\limits_{n\geq 0} \dfrac{f^{(n)}(a)}{n!}(z-a)^n$ tiene radio de convergencia infinito y se verifica que:
            \begin{gather*}
                f(z) = \sum\limits_{n=0}^\infty \frac{f^{(n)}(a)}{n!} (z-a)^n \hspace{1cm} \forall z \in \bb{C}
            \end{gather*}
            \item Si $\Omega \neq \bb{C}$ y para cada $a\in \Omega$ tomamos $R_a = d(a,\bb{C}\setminus \Omega)$, la serie $\sum\limits_{n\geq 0} \dfrac{f^{(n)}(a)}{n!}(z-a)^n$ tiene radio de convergencia mayor o igual que $R_a$ y se verifica que 
            \begin{gather*}
                f(z) = \sum\limits_{n=0}^\infty \frac{f^{(n)}(a)}{n!} (z-a)^n \hspace{1cm} \forall z \in D(a, R_a)
            \end{gather*}
        \end{enumerate}
    \end{teo}

    \begin{prop}[\textbf{Teorema de Cauchy para las derivadas}] Sean $\Omega$ un abierto de $\bb{C}$ y $f\in \cc{H}(\Omega)$. Dado $a\in \Omega$, sea $r\in \bb{R}^+$ tal que $\overline{D}(a,r)\subset \Omega$. Se tiene entonces:
        \begin{gather*}
            f^{(k)} (z) = \frac{k!}{2\pi i } \int_{C(a,r)} \frac{f(w)}{(w-z)^{k+1}} dw \hspace{1cm} \forall z \in D(a,r),\ \forall k \in \bb{N}\cup \{0\}
        \end{gather*}
    \end{prop}

    \begin{teo}[\textbf{Teorema de extensión de Riemann}]
        Sean $\Omega$ un abierto de $\bb{C}$, $z_0 \in \Omega$ y $f\in \cc{H}(\Omega\setminus \{z_0\})$. Las siguientes afirmaciones son equivalentes:
        \begin{enumerate}
            \item Existe $g\in \cc{H}(\Omega)$ tal que $g(z) = f(z)$ para todo $z\in \Omega \setminus \{z_0\}$.
            \item $f$ tiene límite en el punto $z_0$.
            \item Existen $\delta,M>0$ tales que $|f(z)|\leq M$ para todo $z \in \Omega$ que verifique $0 < |z - z_0| < \delta$.
            \item $\lim\limits_{z \to z_0} (z - z_0) f(z) = 0$.
        \end{enumerate}
    \end{teo}

    \section{Tema 9}

    \begin{prop}[\textbf{Desigualdades de Cauchy}] Sean $\Omega$ un abierto de $\bb{C}$, $f\in \cc{H}(\Omega)$ y $a\in \Omega$. Dado $r\in \bb{R}^+$ tal que $\overline{D}(a,r)\subset \Omega$, sea $M(f,a,r) = \max \{|f(z)|: z\in C(a,r)^*\}$. Se tiene entonces:
    \begin{gather*}
        \frac{|f^{(n)}(a)|}{n!} \leq \frac{M(f,a,r)}{r^n} \hspace{1cm} \forall n \in \bb{N}\cup \{0\}
    \end{gather*}       
    \end{prop}

    \begin{teo}[\textbf{Teorema de Liouville}] Toda función entera y acotada es constante. De hecho, la imagen de cualquier función entera no constante es un conjunto denso en $\bb{C}$, es decir, para $f\in \cc{H}(\bb{C})$ tal que $\exists M\in \bb{R}^+$ de forma que $|f(z)| \leq M$ $\forall z \in \bb{C}$, entonces se tiene que $\overline{Im(f)} = \bb{C}$.
    \end{teo}

    \begin{teo}[\textbf{Teorema fundamental de Álgebra}] El cuerpo $\bb{C}$ es algebraicamente cerrado, es decir, si $P$ es un polinomio con coeficientes complejos, no constante, existe $z\in \bb{C}$ tal que $P(z) = 0$.
    \end{teo}

    \begin{prop}[\textbf{Principio de identidad para funciones holomorfas}] Sea $\Omega$ un dominio y $f,g\in \cc{H}(\Omega)$. Si $A$ es un subconjunto de $\Omega$ tal que $f(a) = g(a)$ para todo $a\in A$, y $A' \cap \Omega \neq \emptyset$, entonces $f$ y $g$ son idénticas: $f(z)=g(z)$ para todo $z\in \Omega$.
    \end{prop}

    \section{Tema 10}

    \begin{teo}[\textbf{Teorema de convergencia de Weierstrass}] Sea $\Omega$ un abierto no vacío de $\bb{C}$ y, para cada $n\in \bb{N}$, sea $f_n \in \cc{H}(\Omega)$. Supongamos que la sucesión $\{f_n\}$ converge uniformemente en cada subconjunto compacto de $\Omega$ a una función $f:\Omega \to \bb{C}$, en particular:
    \begin{gather*}
        f(z) = \lim_{n\to \infty} f_n(z) \hspace{1cm} \forall z \in \Omega
    \end{gather*}
    Entonces $f\in \cc{H}(\Omega)$ y, para cada $k\in \bb{N}$, se tiene que la sucesión $\{f^{(k)}_n\}$ de las k-ésimas derivadas, converge a la derivada k-ésima $f^{(k)}$, uniformemente en cada subconjunto compacto de $\Omega$, en particular:
    \begin{gather*}
        f^{(k)} (z) = \lim_{n\to \infty} f_n^{(k)} (z) \hspace{1cm} \forall z \in \Omega,\ \forall k \in \bb{N}
    \end{gather*}
    Este resultado también se puede usar para series
    \end{teo}

    \begin{teo}[\textbf{Holomorfía de la integral dependiente de un parámetro}]
        Sea $\gamma$ u ncamino, $\Omega$ un abierto del plano y $\Phi:\gamma^* \times \Omega \to \bb{C}$ una función continua. Supongamos que, para cada $w\in \gamma^*$, la función $\Phi_w:\Omega \to \bb{C}$ definida por $\Phi_w(z) = \Phi(w,z)$ para todo $z\in \Omega$, es holomorfa en $\Omega$. Entonces, definiendo
        \begin{gather*}
            f(z) = \int_\gamma \Phi(w,z) dw \hspace{1cm} \forall z \in \Omega
        \end{gather*}
        se obtiene una función holomorfa: $f \in \cc{H}(\Omega)$. Además, para cada $k\in \bb{N}$ y cada $z\in \Omega$, la función $w \mapsto \Phi_w ^{(k)}(z)$, de $\gamma^*$ en $\bb{C}$, es continua y se verifica que
        \begin{gather*}
            f^{(k)} (z) = \int_\gamma \Phi_w^{(k)} dw = \int_\gamma \frac{\partial^k \Phi}{\partial z^k}(w,z) \hspace{1cm} \forall z \in \Omega, \forall k \in \bb{N}
        \end{gather*}
    \end{teo}

    \section{Tema 11}

    \begin{prop}[\textbf{Propiedad de la media}]
        Sea $\Gamma$ un abierto de $\bb{C}$ y $f\in \cc{H}(\bb{C})$. Para $a\in \Omega$ y $r\in \bb{R}^+$ tales que $\overline{D}(a,r)\subset \Omega$, se tiene:
        \begin{gather*}
            f(a) = \frac{1}{2\pi} \int_{-\pi}^{\pi} f(a+re^{it}) dt
        \end{gather*}
    \end{prop}

    \begin{teo}[\textbf{Principio del módulo máximo}] Sea $\Omega$ un dominio y $f\in \cc{H}(\Omega)$. Supongamos que $|f|$ tiene un máximo relativo en un punto $a\in \Omega$, es decir, existe $\delta > 0$ tal que $D(a, \delta)\subset \Omega$ y $|f(z)| \leq |f(a)|$ para todo $z\in D(a, \delta)$. Entonces $f$ es constante.
    \end{teo}

    \begin{teo}[\textbf{Principio del módulo mínimo}] Sea $\Omega$ un dominio y $f\in \cc{H}(\Omega)$. Supongamos que $|f|$ tiene un mínimo relativo en un punto $a\in \Omega$ es decir, existe $\delta > 0$ tal que $D(a, \delta)\subset \Omega$ y $|f(z)| \geq |f(a)|$ para todo $z\in D(a, \delta)$. Entonces, o bien $f(a)=0$, o bien $f$ es constante.
    \end{teo}

    \begin{teo}[\textbf{Teorema de la función inversa global}] Sea $U$ un dominio y $f\in \cc{H}(U)$ una función inyectiva. Entonces $V=f(U)$ es un dominio y $f^{-1}\in \cc{H}(V)$ con 
    \begin{gather*}
        (f^{-1})'(f(z)) = \frac{1}{f'(z)} \hspace{1cm} \forall z \in U
    \end{gather*}
    \end{teo}

    \section{Tema 12}

    \begin{teo}[\textbf{Forma general del teorema de Cauchy y la fórmula integral de Cauchy}] Sea $\Omega$ un abierto del plano y $\Gamma$ un ciclo en $\Omega$, nul-homólogo con respecto a $\Omega$. Para toda función $f\in \cc{H}(\Omega)$ se tiene:
    \begin{enumerate}
        \item $\displaystyle Ind_\Gamma (z)f(z) = \frac{1}{2\pi i} \int_\Gamma \frac{f(w)}{w-z} dw$ \hspace{1cm} $\forall z \in \Omega \setminus \Gamma^*$
        \item $\displaystyle \int_\Gamma f(w) dw = 0$
    \end{enumerate}
    \end{teo}

    \section{Tema 13}

    \begin{teo}[\textbf{Desarrolo en serie de Laurent}] Sea $\Gamma = A(a;r,R)$ un anillo abierto arbitrario y $f\in \cc{H}(\Omega)$. Entonces exite una única serie de Laurent no trivial $\sum\limits_{n \in \bb{Z}}c_n(z-a)^n$, cuyo anillo de convergencia contiene a $\Omega$, que verifica:
    \begin{gather*}
        f(z) = \sum\limits_{n=-\infty}^{+\infty} c_n(z-a)^n \hspace{1cm} \forall z \in \Omega
    \end{gather*}
    De hecho, para cualquier $\rho\in \bb{R}^+$ que verifique $r < \rho < R$, se tiene;
    \begin{gather*}
        c_n = \frac{1}{2\pi i} \int_{C(a, \rho)} \frac{f(w)}{(w-a)^{n+1}}dw \hspace{1cm} \forall n \in \bb{Z}
    \end{gather*}
    \end{teo}

    \begin{prop}[\textbf{Caracterización de los puntos regulares}]
        Las siguientes afirmaciones son equivalentes:
        \begin{enumerate}
            \item $a$ es un punto regular de $f$.
            \item $c_{-n} = 0 $ para todo $n\in \bb{N}$.
            \item Existe $g\in \cc{H}(\Omega)$ tal que $f(z) = g(z)$ para todo $z \in \Omega \setminus \{a\}$.
            \item $f$ tiene límite en $a$: $\lim\limits_{z\to a } f(z) = w \in \bb{C}$.
            \item Existen $M,\delta\in \bb{R}^+$ tales que $D(a, \delta) \subset \Omega$ y $|f(z)|\leq M$ para todo $z\in D(a,\delta)\setminus \{a\}$.
            \item $\lim\limits_{z\to a} (z-a)f(z) = 0$
        \end{enumerate}
    \end{prop}

    \begin{prop}[\textbf{Caracterización de los polos teniendo en cuenta su orden}] Dado $k \in \bb{N}$, las siguientes afirmaciones son equivalentes:
        \begin{enumerate}
            \item $a$ es un polo de orden $k$ de $f$.
            \item $c_{-k}\neq 0$ y $c_{-n} = 0$ para $n> k$.
            \item $\lim\limits_{z \to a}(z-a)^kf(z) = \alpha \in \bb{C}^*$.
            \item Existe una función $\psi\in \cc{H}(\Omega)$ con $\psi(a) \neq 0$ tal que:
            \begin{gather*}
                f(z) = \frac{\psi(z)}{(z-a)^k} \hspace{1cm} \forall z \in \Omega \setminus \{a\}
            \end{gather*}
        \end{enumerate}
    \end{prop}

    \begin{prop}[\textbf{Caracterización de los polos}]
        La función $f$ tiene un polo en $a$ si y solo si diverge en $a$.
    \end{prop}

    \begin{teo}[\textbf{Teorema de Casorati}] Las siguientes afirmaciones son equivalentes:
    \begin{enumerate}
        \item La función $f$ tiene una singularidad esencial en el punto $a$
        \item Para cada $\delta \in \bb{R}^+$ con $D(a, \delta)\subset \Omega$, el conjunto $f(D(a, \delta)\setminus \{a\})$ es denso en $C$.
        \item Para cada $w\in \bb{C}$ existe una sucesión $\{z_n\}$ de puntos de $\Omega \setminus \{a\}$ tal que $\{z_n\} \to a$ y $\{f(z_n)\}\to w$. También existe una sucesión $\{u_n\}$ de puntos de $\Omega \setminus \{a\}$ tal que $\{u_n\} \to a$ y $\{f(u_n)\} \to \infty$.
    \end{enumerate}
    \end{teo}

    \section{Tema 14}

    \begin{teo}[\textbf{Teorema de los residuos}]
        Sea $\Omega$ un abierto del plano, $A$ un subconjunto de $\Omega$ tal que $A'\cap \Omega = \emptyset$, y $f\in \cc{H}(\Omega \setminus A)$. Sea $\Gamma$ un ciclo en $\Gamma \setminus A$, nul-homólogo con respecto a $\Omega$. Entonces, el conjunto $\{a\in A : Ind_\Gamma(a)\neq 0\}$ es finito y se verifica que
        \begin{gather*}
            \int_\Gamma f(z)dz = 2\pi i \sum\limits_{a\in A} Ind_\Gamma (a) Res(f(z), a)
        \end{gather*}
    \end{teo}

    \begin{teo}[\textbf{Teorema de l'Hôpital para funciones holomorfas}] Sean $a\in \bb{C}$, $R\in \bb{R}^+$ y $f,g\in \cc{H}(D(a, R))$. Supongamos que $f(a)=g(a)=0$ y que $g$ no es idénticamente nula. Entonces existe un $\delta \in ]0,R[$, tal que $g(z)\neq 0$ y $g'(z)\neq 0$ para todo $z\in D(a, \delta)\setminus \{a\}$. Además, se verifica una de las dos afirmaciones siguientes:
    \begin{enumerate}
        \item $\displaystyle \lim\limits_{z\to a} \frac{f(z)}{g(z)} = \lim\limits_{z\to a} \frac{f'(z)}{g'(z)} = \alpha \in \bb{C}$.
        \item $\dfrac{f(z)}{g(z)} \to \infty\  (z \to a)$ y $\dfrac{f'(z)}{g'(z)} \to \infty\ (z \to a)$.
    \end{enumerate}
        
    \end{teo}
    
\end{document}