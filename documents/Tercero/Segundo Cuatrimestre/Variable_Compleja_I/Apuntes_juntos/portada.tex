\documentclass[12pt]{book}

\input{../../../../_assets/preambulo.tex}

\begin{document}
    \begin{titlepage}
        \vspace*{\fill}
        \begin{center}
            {\Large Variable Compleja I}
        \end{center}
        \vspace*{\fill}
    \end{titlepage}
    \thispagestyle{empty}
    \tableofcontents
    \thispagestyle{empty}
    \thispagestyle{empty}
    \newpage
    \thispagestyle{empty}
    \newpage
    \setcounter{page}{1}
    \chapter{Tema 1: Números Complejos}
    \newpage
    \setcounter{page}{9}
    \chapter{Tema 2: Topología del plano complejo}
    \newpage
    \setcounter{page}{19}
    \chapter{Tema 3: Funciones holomorfas}
    \newpage
    \setcounter{page}{31}
    \chapter{Tema 4: Funciones analíticas}
    \newpage
    \setcounter{page}{49}
    \chapter{Tema 5: Funciones elementales}
    \newpage
    \setcounter{page}{67}
    \chapter{Tema 6: Integral curvilínea}
    \newpage
    \setcounter{page}{83}
    \chapter{Tema 7: Teorema local de Cauchy}
    \newpage
    \setcounter{page}{85}
    \chapter{Tema 8: Equivalencia entre analiticidad y holomorfía}
    \newpage
    \setcounter{page}{105}
    \chapter{Tema 9: Ceros de las funciones holomorfas}
    \newpage
    \setcounter{page}{113}
    \chapter{Tema 10: Teorema de Morera y sus consecuencias}
    \newpage
    \setcounter{page}{123}
    \chapter{Tema 11: Comportamiento local de una función holomorfa}
    \newpage
    \setcounter{page}{133}
    \chapter{Tema 12: El teorema general de Cauchy}
    \newpage
    \setcounter{page}{147}
    \chapter{Tema 13: Singularidades}
    \newpage
    \setcounter{page}{163}
    \chapter{Tema 14: Residuos}
\end{document}