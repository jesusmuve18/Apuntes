\documentclass[12pt]{article}

\input{../../../../../_assets/preambulo.tex}
\usepackage{xkeyval} % Para el paso de argumentos

\usepackage{graphicx}

% Definir la carpeta de las imágenes
\graphicspath{{../_assets}{../../_assets}}

% Definir el comando \portada
\makeatletter
\define@key{portada}{titulo}{\def\titulo{#1}}
\define@key{portada}{subtitulo}{\def\subtitulo{#1}}
\define@key{portada}{autor}{\def\autor{#1}}
\define@key{portada}{año}{\def\año{#1}}

\newcommand*{\portada}[1][]{%
    % Definimos las claves y sus valores por defecto
  \setkeys{portada}{%
    titulo=Sin Título,%
    subtitulo=Sin Subtítulo,%
    autor=Autor Desconocido,%
    año=Sin Año, #1}%
    \begin{titlepage}
        \centering
        {\includegraphics[width=0.2\textwidth]{Logo-UGR-Black.png}\par}
        \vspace{1cm}
        {\bfseries\LARGE Universidad de Granada \par}
        \vspace{1cm}
        {\scshape\Large Doble Grado en Ingeniería Informática y Matemáticas \par}
        \vspace{3cm}
        {\scshape\Huge \titulo \par}
        \vspace{3cm}
        {\itshape\Large \subtitulo \par}
        \vfill
        {\Large Autor: \par}
        {\Large \autor \par}
        \vfill
        {\Large \año \par}
    \end{titlepage}%
}

% Mejor tipografía en bloques de código
\usepackage{inconsolata}

\definecolor{codebg}{gray}{0.9}

% Para código en línea
\newcommand{\icode}[1]{\colorbox{codebg}{\texttt{\detokenize{#1}}}}

% Para bloques de código
\lstdefinestyle{mycode}{
  backgroundcolor=\color{codebg},   
  basicstyle=\ttfamily\small,
  numbers=none,               
  frame=none,                 
  breaklines=true,            
  showstringspaces=false,
  aboveskip=1em,   % espacio antes del bloque
  belowskip=1em    % espacio después del bloque
}

\lstnewenvironment{icodeblock}[1][]
{
  \lstset{style=mycode, language=#1}
}
{}

\usepackage{sectsty} % paquete para modificar estilos de secciones
\allsectionsfont{\sffamily}

\newcommand{\img}[1] {
    \begin{center}
        \includegraphics[width=15cm]{./images/img#1.png}
    \end{center}
}


\begin{document}

\sffamily
    \portada[%
        titulo=Ingeniería de Servidores (Prácticas),
        subtitulo=Ejercicio Evaluable docker compose,
        autor=Jesús Muñoz Velasco,
        año=Curso 2024-2025]
        
    \thispagestyle{empty}
    \tableofcontents
    \newpage

    \section{Enunciado}
Instale Docker en su ordenador anfitrión o en una MV y ejecute el contenedor “Hello World” disponible en: \url{https://hub.docker.com/_/hello-world}.

\section{Resolución}

Se ha elegido la instalación en local (ordenador personal)

En primer lugar se ha instalado docker con los siguientes comandos:
\begin{icodeblock}[bash]
for pkg in docker.io docker-doc docker-compose docker-compose-v2 podman-docker containerd runc; do sudo apt-get remove \$pkg; done
\end{icodeblock}
Con esto eliminamos los posibles paquetes conflictivos y no oficiales de docker.

Ahora pasamos a configurar el repositorio `apt`:
\begin{icodeblock}[bash]
# Add Docker's official GPG key:
sudo apt-get update
sudo apt-get install ca-certificates curl
sudo install -m 0755 -d /etc/apt/keyrings
sudo curl -fsSL https://download.docker.com/linux/ubuntu/gpg -o /etc/apt/keyrings/docker.asc
sudo chmod a+r /etc/apt/keyrings/docker.asc

# Add the repository to Apt sources:
echo \
  "deb [arch=\$(dpkg --print-architecture) signed-by=/etc/apt/keyrings/docker.asc] https://download.docker.com/linux/ubuntu \
  \$(. /etc/os-release && echo "\${UBUNTU_CODENAME:-\$VERSION_CODENAME}") stable" | \
  sudo tee /etc/apt/sources.list.d/docker.list > /dev/null
sudo apt-get update
\end{icodeblock}

Una vez hecho esto pasamos finalmente a instalar docker con:
\begin{icodeblock}[bash]
sudo apt-get install docker-ce docker-ce-cli containerd.io docker-buildx-plugin docker-compose-plugin
\end{icodeblock}

Con esto ya estaría instalado y podemos hacer la prueba que se pide:
\begin{icodeblock}[bash]
sudo docker run hello-world
\end{icodeblock}
y obtendremos la siguiente salida:
\begin{icodeblock}[bash]
jesusmuve@jesusmuve-Modern-14-B10MW:~\$ sudo docker run hello-world
Unable to find image 'hello-world:latest' locally
latest: Pulling from library/hello-world
e6590344b1a5: Pull complete 
Digest: sha256:dd01f97f252193ae3210da231b1dca0cffab4aadb3566692d6730bf93f123a48
Status: Downloaded newer image for hello-world:latest

Hello from Docker!
This message shows that your installation appears to be working correctly.

To generate this message, Docker took the following steps:
 1. The Docker client contacted the Docker daemon.
 2. The Docker daemon pulled the "hello-world" image from the Docker Hub.
    (amd64)
 3. The Docker daemon created a new container from that image which runs the
    executable that produces the output you are currently reading.
 4. The Docker daemon streamed that output to the Docker client, which sent it
    to your terminal.

To try something more ambitious, you can run an Ubuntu container with:
 \$ docker run -it ubuntu bash

Share images, automate workflows, and more with a free Docker ID:
 https://hub.docker.com/

For more examples and ideas, visit:
 https://docs.docker.com/get-started/

\end{icodeblock}

Adjunto la captura de pantalla que muestra dicha salida (img0):

\img{0}

\end{document}