\documentclass[12pt]{article}

\input{../../../../../_assets/preambulo.tex}
\usepackage{xkeyval} % Para el paso de argumentos

\usepackage{graphicx}

% Definir la carpeta de las imágenes
\graphicspath{{../_assets}{../../_assets}}

% Definir el comando \portada
\makeatletter
\define@key{portada}{titulo}{\def\titulo{#1}}
\define@key{portada}{subtitulo}{\def\subtitulo{#1}}
\define@key{portada}{autor}{\def\autor{#1}}
\define@key{portada}{año}{\def\año{#1}}

\newcommand*{\portada}[1][]{%
    % Definimos las claves y sus valores por defecto
  \setkeys{portada}{%
    titulo=Sin Título,%
    subtitulo=Sin Subtítulo,%
    autor=Autor Desconocido,%
    año=Sin Año, #1}%
    \begin{titlepage}
        \centering
        {\includegraphics[width=0.2\textwidth]{Logo-UGR-Black.png}\par}
        \vspace{1cm}
        {\bfseries\LARGE Universidad de Granada \par}
        \vspace{1cm}
        {\scshape\Large Doble Grado en Ingeniería Informática y Matemáticas \par}
        \vspace{3cm}
        {\scshape\Huge \titulo \par}
        \vspace{3cm}
        {\itshape\Large \subtitulo \par}
        \vfill
        {\Large Autor: \par}
        {\Large \autor \par}
        \vfill
        {\Large \año \par}
    \end{titlepage}%
}

% Mejor tipografía en bloques de código
\usepackage{inconsolata}

\definecolor{codebg}{gray}{0.9}

% Para código en línea
\newcommand{\icode}[1]{\colorbox{codebg}{\texttt{\detokenize{#1}}}}

% Para bloques de código
\lstdefinestyle{mycode}{
  backgroundcolor=\color{codebg},   
  basicstyle=\ttfamily,      
  numbers=none,               
  frame=none,                 
  breaklines=true,            
  showstringspaces=false,
  aboveskip=1em,   % espacio antes del bloque
  belowskip=1em    % espacio después del bloque
}

\lstnewenvironment{icodeblock}[1][]
{
  \lstset{style=mycode, language=#1}
}
{}

 % Soporte para yaml
\lstdefinelanguage{yml}{
  sensitive=true,
  morecomment=[l]{\#},
  morestring=[b]",
  morestring=[b]',
  morekeywords={true,false,null,yes,no},
  alsoletter={-},
  keywordstyle=\color{blue},
  commentstyle=\color{gray},
  stringstyle=\color{orange}
}

\usepackage{sectsty} % paquete para modificar estilos de secciones
\allsectionsfont{\sffamily}

% Insertar imagenes solo con el número
\newcommand{\img}[1] {
    \begin{center}
        \includegraphics[width=15cm]{./images/img#1.png}
    \end{center}
}


\begin{document}

\sffamily
    \portada[%
        titulo=Ingeniería de Servidores (Prácticas),
        subtitulo=Monitorización de API Web,
        autor=Jesús Muñoz Velasco,
        año=Curso 2024-2025]
        
    \thispagestyle{empty}
    \tableofcontents
    \newpage

    \section{Enunciado}
La aplicación empleada en apartado anterior para la prueba de carga, expone en el path “/metrics” los indicadores de NodeJS para Prometheus. Para más información, el exporter de Prometheus de la API Web se ha generado empleando los componentes estandar: prom-client y express-prom-bundle.\\

Cree un nuevo Dashboard con algunas de las métricas expuestas. Para el dashboard emplee como nombre su nombre y apellidos en CamelCase seguido del sufijo API. Por ejemplo, anaTorrentRamonetAPI. Todos los paneles creados se presentarán con un título que contenga las iniciales del alumno/a. Siguiendo con el ejemplo anterior: \%Memoria (ATR).\\

Cree monitores para las siguientes métricas:

\begin{itemize}
    \item Tiempos de respuesta de los endpoints de la API \\
    (\icode{http_request_duration_seconds_bucket})
    \item Memoria disponible (\icode{nodejs_heap_size_total_bytes}) vs la usada actualmente (\icode{nodejs_heap_size_used_bytes})
    \item Uso de CPU (\icode{process_cpu_seconds_total})
\end{itemize}

Realice una memoria de prácticas en la que se ponga de manifiesto la ejecución de la prueba de carga diseñada para Jmeter y se aprecie el efecto de la misma en los monitores anteriormente descritos.


\newpage

\section{Resolución}

\subsection{Configuración Inicial}

Para comenzar el ejercicio se ha copiado el directorio del ejercicio anterior y se ha metido en una carpeta llamada \icode{Grafana_Prometheus_docker} en el cual se han cambiado algunos archivos de configuración para que prometheus pueda monitorizar la API Web. La red de directorios resultante ha sido la siguiente

\begin{icodeblock}[bash]
Grafana_Prometheus_docker
  |-grafana_data
  |-prometheus_data
  |-docker-compose.yml
  |-prometheus.yml
\end{icodeblock}

Donde el archivo \icode{docker_compose.yml} contiene la siguiente configuración:

\begin{icodeblock}[yml]
---
version: "3"
services:
  prometheus:
    image: prom/prometheus:v2.50.0
    extra_hosts:
      - "host.docker.internal:host-gateway"
    ports:
      - 9090:9090
    volumes:
      - ./prometheus_data:/prometheus
      - ./prometheus.yml:/etc/prometheus/prometheus.yml
    command:
    - "--config.file=/etc/prometheus/prometheus.yml"
  grafana:
    image: grafana/grafana:9.1.0
    ports:
    - 4000:3000
    volumes:
    - ./grafana_data:/var/lib/grafana
    depends_on:
    - prometheus
\end{icodeblock}

Y al archivo \icode{prometheus.yml} se le ha añadido un job para poder monitorizar la API:
\begin{icodeblock}[yml]
---
global:
  scrape_interval: 5s

scrape_configs:
  - job_name: "prometheus_service"
    static_configs:
      - targets: ["prometheus:9090"]
  
  - job_name: "API_Web"
    static_configs:
      - targets: ['host.docker.internal:3000'] # Acceso a API Web

\end{icodeblock}

De esta forma podemos conectar los componentes necesarios para la realización del ejercicio. Si volvemos al directorio de jMeter y lanzamos el docker con la API:

\begin{icodeblock}[bash]
sudo docker compose up
\end{icodeblock}

Y hacemos lo mismo en el directorio \icode{Grafana_Prometheus_docker} para activar grafana y prometheus tenemos ya la configuración terminada.
Se habilitan los siguientes puertos:
- \icode{localhost:3000}: API de la Web de la ETSIIT
- \icode{localhost:9090}: Prometheus monitorizando la API de la Web de la ETSIIT.
- \icode{localhost:4000}: Grafana

Y ya podremos comenzar con el ejercicio

\subsection{Monitorización}

En primer lugar crearemos los paneles que se piden en Grafana. Para ello abrimos Grafana y le especificamos en la configuración que coja datos de Prometheus (especificando el \icode{Data Source}).

Creamos ahora un nuevo Dashboard (en mi caso lo cree vacío solo para añadir los paneles requeridos). Dentro del Dashboard pasamos a crear los siguientes paneles:

\subsubsection{Tiempo de Uso de la CPU}
En el código del Query añadimos:

\begin{icodeblock}
rate(process_cpu_seconds_total[1m])
\end{icodeblock}

Y conseguimos el tiempo de uso de la CPU por segundo. Tocando un poco las leyendas y los estilos, poniéndolo de tipo Stat obtenemos el siguiente resultado (img37):
\img{37}

\subsubsection{Memoria disponible}
Añadimos ahora dos Querys y ponemos:

\begin{icodeblock}[bash]
nodejs_heap_size_total_bytes # En la primera
nodejs_heap_size_used_bytes  # En la segunda
\end{icodeblock}

Y conseguimos dos líneas temporales que indican la memoria usada y la memoria total en cada caso (img38).

\img{38}
\subsubsection{Tiempos de respuesta de los endpoints de la API}
Añadimos ahora:
\begin{icodeblock}[bash]
histogram_quantile(0.95, sum(rate(prometheus_http_request_duration_seconds_bucket[\$__rate_interval])) by (le))
\end{icodeblock}
(img39).

\img{39}
Ya podemos pasar a hacer la prueba de jMeter. Como jMeter estaba configurado para acceder a la API a través de \icode{localhost:3000} no tendremos que modificar nada en su configuración, simplemente ejecutar la prueba de carga.

Al ejecutarlo tenemos inicialmente una situación como la siguiente (img40):
\img{40}
donde todo el sistema monitorizado parece en un estado de uso "bajo".

Al ejecutar la prueba tenemos el siguiente resultado (img41):
\img{41}

Donde se aprecia perfectamente un pico en las medidas tanto de tiempo de uso de CPU como de memoria usada y de tiempo de acceso.

\end{document}