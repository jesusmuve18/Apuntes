\documentclass[12pt]{book}

\input{../_assets/preambulo.tex}
\usepackage{xkeyval} % Para el paso de argumentos

\usepackage{graphicx}

% Definir la carpeta de las imágenes
\graphicspath{{../_assets}{../../_assets}}

% Definir el comando \portada
\makeatletter
\define@key{portada}{titulo}{\def\titulo{#1}}
\define@key{portada}{subtitulo}{\def\subtitulo{#1}}
\define@key{portada}{autor}{\def\autor{#1}}
\define@key{portada}{año}{\def\año{#1}}

\newcommand*{\portada}[1][]{%
    % Definimos las claves y sus valores por defecto
  \setkeys{portada}{%
    titulo=Sin Título,%
    subtitulo=Sin Subtítulo,%
    autor=Autor Desconocido,%
    año=Sin Año, #1}%
    \begin{titlepage}
        \centering
        {\includegraphics[width=0.2\textwidth]{Logo-UGR-Black.png}\par}
        \vspace{1cm}
        {\bfseries\LARGE Universidad de Granada \par}
        \vspace{1cm}
        {\scshape\Large Doble Grado en Ingeniería Informática y Matemáticas \par}
        \vspace{3cm}
        {\scshape\Huge \titulo \par}
        \vspace{3cm}
        {\itshape\Large \subtitulo \par}
        \vfill
        {\Large Autor: \par}
        {\Large \autor \par}
        \vfill
        {\Large \año \par}
    \end{titlepage}%
}

\begin{document}
    \portada[%
        titulo=Ecuaciones Diferenciales I,
        subtitulo=Primera Prueba de Clase,
        autor=Jesús Muñoz Velasco,
        año=Curso 2024-2025]

    \begin{ejercicio}
        En el plano con coordenadas $(A,B)$ se considera la ecuación
        \begin{gather*}
            A^3-cos(AB)=0
        \end{gather*}
        ¿Es posible encontrar una función $\varphi:I \to \bb{R},B \mapsto \pi(B)$ con $\varphi(0)=1$ y de manera que se cumpla la ecuación para cada $(A,B)$ con $A=\varphi(B), B\in I$? $I=]-\delta, \delta[$ para algún $\delta>0$
    \end{ejercicio}

    \begin{ejercicio}
        Se considera la familia uniparamétrica de curvas
        \begin{gather*}
            y = \frac{x^2}{2}+c,\ \ c\in \bb{R}
        \end{gather*}
        Encuentra la familia de trayectorias ortogonales y dibuja las dos familias en un plano común con coordenadas $(x,y)$.
    \end{ejercicio}

    \begin{ejercicio}
        Encuentra la solución del problema de valores iniciales 
        \begin{gather*}
            x'=\frac{1+x^2}{1+t^2},\ \ x(0)=1.
        \end{gather*}
        ¿En qué intervalo está definida?
    \end{ejercicio}
    \begin{ejercicio}
        Se considera la transformación
        \begin{gather*}
            \phi:D \to \bb{R}^2,\ \ \phi(t,x)=(e^tx, \arctan x)
        \end{gather*}
        donde $D=\bb{R}\times ]0,\infty[$. Se pide:
        \begin{enumerate}
            \item[a)] Describe el conjunto $D_1=\varphi(D)$ y prueba que $\varphi$ es un difeomorfismo entre $D$ y $D_1$.
            \item[b)] Dada una ecuación $x'=f(t,x)$ con $f:D \to \bb{R}$, ¿qué condiciones hay que imponer para que se pueda asegurar que el cambio $(s,y)=\varphi(t,x)$ es admisible?
        \end{enumerate}
    \end{ejercicio}

    \begin{ejercicio}
        Dada una función $\phi:\bb{R} \to \bb{R}$ de clase $C^1$, se considera el cambio de variable
        \begin{gather*}
            s=t+\phi(x),\ \ y=x
        \end{gather*}
        \begin{enumerate}
            \item[a)] Prueba que $\varphi:\bb{R}^2 \to \bb{R}^2$, $\varphi(t,x)=(s,y)$ es un difeomorfismo.
            \item[b)] Encuentra una función $\phi$ en las condiciones anteriores que permita transformar la ecuación $x'=x$ en la ecuación
            \begin{gather*}
                \frac{dy}{ds}=\frac{y}{1+y\cos y}
            \end{gather*}
            Describe los dominios sobre los que este cambio es admisible.
        \end{enumerate}
    \end{ejercicio}
\end{document}