\documentclass[12pt]{article}

\input{../_assets/preambulo.tex}
\usepackage{xkeyval} % Para el paso de argumentos

\usepackage{graphicx}

% Definir la carpeta de las imágenes
\graphicspath{{../_assets}{../../_assets}}

% Definir el comando \portada
\makeatletter
\define@key{portada}{titulo}{\def\titulo{#1}}
\define@key{portada}{subtitulo}{\def\subtitulo{#1}}
\define@key{portada}{autor}{\def\autor{#1}}
\define@key{portada}{año}{\def\año{#1}}

\newcommand*{\portada}[1][]{%
    % Definimos las claves y sus valores por defecto
  \setkeys{portada}{%
    titulo=Sin Título,%
    subtitulo=Sin Subtítulo,%
    autor=Autor Desconocido,%
    año=Sin Año, #1}%
    \begin{titlepage}
        \centering
        {\includegraphics[width=0.2\textwidth]{Logo-UGR-Black.png}\par}
        \vspace{1cm}
        {\bfseries\LARGE Universidad de Granada \par}
        \vspace{1cm}
        {\scshape\Large Doble Grado en Ingeniería Informática y Matemáticas \par}
        \vspace{3cm}
        {\scshape\Huge \titulo \par}
        \vspace{3cm}
        {\itshape\Large \subtitulo \par}
        \vfill
        {\Large Autor: \par}
        {\Large \autor \par}
        \vfill
        {\Large \año \par}
    \end{titlepage}%
}
\usepackage{ wasysym }

\begin{document}

\portada[%
        titulo=Fundamentos de Bases de Datos,
        subtitulo=Ejercicio 7 (AR-II) y Ejercicio 3.7 (SQL),
        autor=Jesús Muñoz Velasco,
        año=Curso 2024-2025]

\setcounter{ejercicio}{6}

\begin{ejercicio}(AR-II)
    Encontrar los códigos de los proyectos que usan una pieza que vende S1.
    \begin{align*}
        \pi_{codpj}(\sigma_{codpro=S1}(ventas))
    \end{align*}
\end{ejercicio}

\setcounter{ejercicio}{2}
\begin{ejercicio}\hspace{-0.15cm}\textbf{7.}(SQL) Resolver la consulta del ejemplo 3.8 utilizando el operador $\cap$.
    \endsquare
Recordemos que el ejemplo 3.8 era el siguiente:

Ciudades donde viven proveedores con status mayor 
de 2 en las que no se fabrica la pieza 'P1'.


\begin{minted}{sql}
(SELECT DISTINCT ciudad
FROM proveedor
WHERE status>2)
MINUS
(SELECT DISTINCT ciudad
FROM pieza
WHERE codpie='P1');
\end{minted}
La solución propuesta inicialmente utlizando la intersección es la siguiente:
\begin{minted}{sql}
SELECT DISTINCT ciudad
FROM proveedor
WHERE status > 2
AND ciudad NOT IN (
    SELECT DISTINCT ciudad
    FROM proveedor
    WHERE status > 2
    INTERSECT
    SELECT DISTINCT ciudad
    FROM pieza
    WHERE codpie = 'P1'
);
\end{minted}
En clase se trabajaron dos soluciones alternativas adicionales, l aprimera con el producto cartesiano y una segunda que evita incluso el uso de la intersección:
\begin{minted}{sql}
--Segunda opcion
(SELECT DISTINCT proveedor.ciudad
FROM proveedor
WHERE proveedor.status>2
INTERSECT
SELECT proveedor.ciudad
FROM proveedor, pieza
WHERE pieza.codpie='P1' AND proveedor.ciudad!=pieza.ciudad);

--Tercera opcion (sin usar intersect)
SELECT proveedor.ciudad
FROM proveedor, pieza
WHERE pieza.codpie='P1' 
    AND proveedor.ciudad!=pieza.ciudad 
    AND proveedor.status>2;

\end{minted}
\end{ejercicio}
    

\end{document}