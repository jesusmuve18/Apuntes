\documentclass[12pt]{article}

\input{../_assets/preambulo.tex}
\usepackage{xkeyval} % Para el paso de argumentos

\usepackage{graphicx}

% Definir la carpeta de las imágenes
\graphicspath{{../_assets}{../../_assets}}

% Definir el comando \portada
\makeatletter
\define@key{portada}{titulo}{\def\titulo{#1}}
\define@key{portada}{subtitulo}{\def\subtitulo{#1}}
\define@key{portada}{autor}{\def\autor{#1}}
\define@key{portada}{año}{\def\año{#1}}

\newcommand*{\portada}[1][]{%
    % Definimos las claves y sus valores por defecto
  \setkeys{portada}{%
    titulo=Sin Título,%
    subtitulo=Sin Subtítulo,%
    autor=Autor Desconocido,%
    año=Sin Año, #1}%
    \begin{titlepage}
        \centering
        {\includegraphics[width=0.2\textwidth]{Logo-UGR-Black.png}\par}
        \vspace{1cm}
        {\bfseries\LARGE Universidad de Granada \par}
        \vspace{1cm}
        {\scshape\Large Doble Grado en Ingeniería Informática y Matemáticas \par}
        \vspace{3cm}
        {\scshape\Huge \titulo \par}
        \vspace{3cm}
        {\itshape\Large \subtitulo \par}
        \vfill
        {\Large Autor: \par}
        {\Large \autor \par}
        \vfill
        {\Large \año \par}
    \end{titlepage}%
}
\usepackage{ wasysym }

\begin{document}

\portada[%
        titulo=Fundamentos de Bases de Datos,
        subtitulo=Ejercicio 3.24 (SQL),
        autor=Jesús Muñoz Velasco,
        año=Curso 2024-2025]

\setcounter{ejercicio}{2}
\begin{ejercicio}\hspace{-0.15cm}\textbf{24.}(SQL) Encontrar los códigos de las piezas suministradas a todos los proyectos
    localizados en Londres.\\

    La resolución de este ejercicio consistirá en resolver con SQL la siguiente consulta (en Álgebra Relacional):
    \begin{align*}
        \pi_{codpie}(\pi_{codpie, codpro}(ventas) \% \pi_{codpro}(\sigma_{ciudad='Londres'}(proyecto))
    \end{align*}

    Para ello se ofrecen las siguientes solucionas (cada una resuelta con un patrón de los que aparecen en el cuadernillo de prácticas en la sección de división):

    \begin{minted}{sql}
--Aproximacion Algebra Relacional
(SELECT codpie FROM pieza)
MINUS
(SELECT codpie
    FROM ((SELECT pieza.codpie, proyecto.codpj
           FROM pieza, proyecto
           WHERE proyecto.ciudad='Londres')
           MINUS
           (SELECT codpie, codpj
           FROM ventas))
);

--Aproximacion Calculo Relacional
SELECT codpie
FROM pieza
WHERE NOT EXISTS (
    SELECT *
    FROM proyecto
    WHERE ciudad='Londres'
        AND NOT EXISTS (
            SELECT *
            FROM ventas
            WHERE pieza.codpie= ventas.codpie
                AND codpj=proyecto.codpj));

--Aproximacion mixta
SELECT codpie 
FROM pieza
WHERE NOT EXISTS (
       (SELECT codpj 
        FROM proyecto
        WHERE proyecto.ciudad='Londres')
    MINUS
       (SELECT codpj 
        FROM ventas 
        WHERE ventas.codpie=pieza.codpie)
);
    \end{minted}


\end{ejercicio}
    

\end{document}