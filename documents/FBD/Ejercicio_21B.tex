\documentclass[12pt]{article}

\input{../_assets/preambulo.tex}
\usepackage{xkeyval} % Para el paso de argumentos

\usepackage{graphicx}

% Definir la carpeta de las imágenes
\graphicspath{{../_assets}{../../_assets}}

% Definir el comando \portada
\makeatletter
\define@key{portada}{titulo}{\def\titulo{#1}}
\define@key{portada}{subtitulo}{\def\subtitulo{#1}}
\define@key{portada}{autor}{\def\autor{#1}}
\define@key{portada}{año}{\def\año{#1}}

\newcommand*{\portada}[1][]{%
    % Definimos las claves y sus valores por defecto
  \setkeys{portada}{%
    titulo=Sin Título,%
    subtitulo=Sin Subtítulo,%
    autor=Autor Desconocido,%
    año=Sin Año, #1}%
    \begin{titlepage}
        \centering
        {\includegraphics[width=0.2\textwidth]{Logo-UGR-Black.png}\par}
        \vspace{1cm}
        {\bfseries\LARGE Universidad de Granada \par}
        \vspace{1cm}
        {\scshape\Large Doble Grado en Ingeniería Informática y Matemáticas \par}
        \vspace{3cm}
        {\scshape\Huge \titulo \par}
        \vspace{3cm}
        {\itshape\Large \subtitulo \par}
        \vfill
        {\Large Autor: \par}
        {\Large \autor \par}
        \vfill
        {\Large \año \par}
    \end{titlepage}%
}

\begin{document}

\portada[%
        titulo=Fundamentos de Bases de Datos,
        subtitulo=Ejercicio 21b del Seminario 2,
        autor=Jesús Muñoz Velasco,
        año=Curso 2024-2025]

\setcounter{ejercicio}{21}

\begin{ejercicio}
    Ponga un ejemplo original de restricciones semánticas y elabore el correspondiente esquema E/R para cada una de las siguientes situaciones:

    \begin{enumerate}
        \item[b)] Una jerarquía con obligatoriedad y exclusividad.
        \endsquare
        En clase el diagrama que quedó finalmente en este apartado fue el siguiente:
        \begin{center}
            \begin{tikzpicture}
                \draw (-1.5,0) rectangle (1.5,1);
                \node at (0,0.5) {Animal};

                \draw (0.05,0) -- (0.05,-1);
                \draw (-0.05,0) -- (-0.05,-1);

                \node at (1,-0.5) {Disjunta};

                \draw (-2,-2.5) -- (-2,-1.5) -- (2,-1.5) -- (2,-2.5);

                \filldraw[fill=white] (-1,-1) -- (1,-1) -- (0,-2) -- cycle;
                \node at (0,-1.25) {Es un};

                \draw (-3.5,-2.5) rectangle (-0.5,-3.5);
                \node at (-2,-3) {Vertebrado};

                \draw (3.5,-2.5) rectangle (0.5,-3.5);
                \node at (2,-3) {Invertebrado};

            \end{tikzpicture}
        \end{center}

        Para poder crear las tablas del modelo relacional necesitamos añadir algún atributo a este diagrama. Podríamos plantearlo así:

        \begin{center}
            \begin{tikzpicture}

                % Entidad Animal
                \draw (-1.5,0) rectangle (1.5,1);
                \node at (0,0.5) {Animal};

                % Atributos Animal
                \draw (1.5, 0.25) -- (2, 0.25);
                \filldraw[fill=white] (2,0.25) circle (2pt) node[right] {Atr\_Animal};

                \draw (1.5, 0.75) -- (2, 0.75);
                \filldraw (2,0.75) circle (2pt) node[right] {Cod\_Especie};

                % Línea obligatoria
                \draw (0.05,0) -- (0.05,-1);
                \draw (-0.05,0) -- (-0.05,-1);
                \node at (1,-0.5) {Disjunta};

                % Líneas jerarquía
                \draw (-2,-2.5) -- (-2,-1.5) -- (2,-1.5) -- (2,-2.5);

                % Triángulo jerarquía
                \filldraw[fill=white] (-1,-1) -- (1,-1) -- (0,-2) -- cycle;
                \node at (0,-1.25) {Es un};

                % Entidad Vertebrado
                \draw (-3.5,-2.5) rectangle (-0.5,-3.5);
                \node at (-2,-3) {Vertebrado};

                % Atributos Vertebrado
                \draw (-2,-3.5) -- (-2,-4);
                \filldraw[fill=white] (-2,-4) circle (2pt) node[below] {Atr\_Vertebrado};

                % Entidad Invertebrado
                \draw (3.5,-2.5) rectangle (0.5,-3.5);
                \node at (2,-3) {Invertebrado};

                % Atributos Invertebrado
                \draw (2,-3.5) -- (2,-4);
                \filldraw[fill=white] (2,-4) circle (2pt) node[below] {Atr\_Invertebrado};

            \end{tikzpicture}
        \end{center}
        Con este diagrama nos quedarían las siguientes tablas:

        \begin{center}
            \begin{tikzpicture}
                % Desactiva los caracteres conflictivos
                \shorthandoff{>} % Para poner puntas de flecha

                \node at (0,0) {Animal(\underline{Cod\_Especie}, Atr\_Animal)};
                \node at (-0.5,-0.5) {CP};

                \node at (-3,-2) {Vertebrado({\underline{Cod\_Especie}}, Atr\_Vertebrado)};
                \node at (-3.5,-2.5) {CP};

                \draw (-4.5,-1.6) -- (-2.5, -1.6);
                \node at (-3.5,-1.4) {CE};

                \node at (3,-4) {Invertebrado(\underline{Cod\_Especie}, Atr\_Invertebrado)};
                \node at (2.5,-4.5) {CP};

                \draw (3.5,-3.6) -- (1.5,-3.6);
                \node at (2.5,-3.4) {CE};

                \draw[->] (-3,-1.25) -- (-0.9,-0.5);
                \draw[->] (2.25,-3) -- (-0.1,-0.5);
            \end{tikzpicture}
        \end{center}
        Estas tablas no admiten fusión por tratarse de una jerarquía.
    \end{enumerate}
\end{ejercicio}

\end{document}