\section{Espacios de Hilbert}

Recordemos que un espacio de Hilbert es un par $(H, (\cdot, \cdot))$ donde $H$ es un espacio vectorial y $(\cdot, \cdot)$ es una función bilineal simétrica y definida positiva.

\begin{prop}
    Si $H$ es prehilbertiano entonces se tiene:
    \begin{enumerate}
        \item Se cumple la Desigualdad de Cauchy-Schwarz, es decir
        \begin{gather*}
            |(u,v)| \leq \|u\| \cdot \|v\|, \ \ \ \forall u,v\in H
        \end{gather*}
        \item Se verifica la desigualdad del paralelogramo
        \begin{gather*}
            \left\| \frac{u+v}{2} \right\|^2 + \left\| \frac{u-v}{2} \right\|^2 = \frac{1}{2}\left(\|u\|^2 + \|v\|^2\right), \ \ \ \forall u,v\in H
        \end{gather*}
    \end{enumerate}
\end{prop}

\begin{teo}[Teorema de la Proyección]
    Supongamos que $H$ es un espacio Hilbertiano y $\emptyset \neq K \subset H$ un conjunto convexo y cerrado, entonces $\forall f\in H$ $\exists_1 u\in K$ tal que $\|f-u\|=dist(f,K)$. Además, dicho $u$ está caracterizado por:
    \begin{gather*}
        \left\{
            \begin{array}{l}
                u\in K\\
                (f-u, v-u)\leq 0$ \ \ $\forall v \in K
            \end{array}
        \right.
    \end{gather*}
    Notaremos a dicho $u$ por $P_Kf$ y \textbf{diremos que es la proyección de $f$ sobre $K$}

    \begin{proof}
        En primer lugar tendremos que ver que $d(f,K)=\inf\{\|f-v\|: v\in K\}$ existe y se alcanza. Al ser un ínfimo de cantidades positivas sabemos que existe y nos quedará ver que se alcanza.

        Por definición de ínfimo tenemos que 
        \begin{gather*}
            \exists\{v_n\}\subset K \text{ tal que } \|f-v_n\|\to d
        \end{gather*}
        Aplicando la desigualdad del paralelogramo para $u=f-v_n$ y $v=f-v_m$, con $n,m\in \bb{N}$
        \begin{gather*}
            \left\| \frac{f-v_n + f-v_m}{2} \right\|^2 + \left\| \frac{f-v_n-(f-v_m)}{2} \right\|^2 = \frac{1}{2}\left(\|f-v_n\|^2 + \|f-v_m\|^2\right)\\
            \left\| f - \frac{v_n+v_m}{2} \right\|^2 + \left\| \frac{v_m - v_n}{2} \right\|^2 = \frac{1}{2}\left(\|f-v_n\|^2 + \|f-v_m\|^2\right)\\
            \frac{\left\| v_m-v_n \right\|^2}{4} = \frac{1}{2}\left(\|f-v_n\|^2 + \|f-v_m\|^2\right) - \left\| f - \frac{v_n+v_m}{2} \right\|^2\\
            \left\| v_m-v_n \right\|^2 = 2\left(\|f-v_n\|^2 + \|f-v_m\|^2\right) - 4\left\| f - \frac{v_n+v_m}{2} \right\|^2
        \end{gather*}
        Como $K$ es convexo y $v_n,v_m\in K$ tendremos que $d\frac{v_n+v_m}{2}\in K$ y además $\left\|f - \dfrac{v_n+v_m}{2}\right\|\geq d$ por lo que tenemos 
        \begin{gather*}
            \left\| v_m-v_n \right\|^2 = 2\left(\|f-v_n\|^2 + \|f-v_m\|^2\right) - 4d^2
        \end{gather*}
        Cuando $n\to \infty$ tenemos que $\|f-v_n\|\to d$ y $\|f-v_m\|\to d$ por lo que el término de la derecha tenderá a 0 cuando $n,m\to\infty$. Esto significa que la sucesión $\{v_n\}$ es de Cauchy.

        Como $H$ es de Hilbert, en particular es completo por lo que sabemos que $\{v_n\}\to u$ en $(H, (\cdot, \cdot))$.

        Como además $\{v_n\}\subset K$ y $K$ es cerrado, el límite $u\in K$. Tendremos que
        \begin{gather*}
            d = \lim_{n\to \infty} \|f-v_n\| = \|f-u\|
        \end{gather*}
        Y tendremos probada la existencia de $u$.\\

        Veamos ahora la equivalencia entre la primera y la segunda parte del teorema, es decir
        \begin{gather*}
            \left.
            \begin{array}{l}
                u\in K\\
                \|f-u\|=dist(f,K)
            \end{array}
            \right\} \sii
            \left\{
            \begin{array}{l}
                u\in K\\
                (f-u, v-u)\leq 0$ \ \ $\forall v \in K
            \end{array}
            \right.
        \end{gather*}
        Veamos las dos implicaciones:
        \begin{itemize}
            \item[$\Rightarrow$)] Supongamos que $u\in K$ y sabemos que $\|f-u\|\leq \|f-v\|$ para todo $v\in K$. Tomamos ahora $w\in K$ y consideramos el segmento que une $u$ con $w$. Entonces $\forall w\in K$ y $\forall t \in [0,1]$, al ser $K$ convexo tendremos que
            \begin{gather*}
                (1-t)u + tw \in K\ \  \text{ y }\ \ \|f-u\|^2 \leq \|f-(1-t)u-tw\|^2
            \end{gather*}
            Aplicando la bilinealidad podemos reescribir esta última expresión como 
            \begin{align*}
                \|f-(1-t)u-tw\|^2 &= (f-(1-t)u-tw,f-(1-t)u-tw) =\\
                &=\|f-u\|^2 + t^2\|w-u\|^2-2t(f-u,w-u)
            \end{align*}
            Sustituyendo en la expresión que teníamos anteriormente nos queda que:
            \begin{gather*}
                0\leq t^2\|w-u\|^2-2t(f-u,w-u) \ \ \ \forall t \in (0,1]
            \end{gather*}
            Al dividir entre $t$ nos queda
            \begin{gather*}
                0\leq t\|w-u\|^2-2(f-u,w-u) \ \ \ \forall t \in (0,1]
            \end{gather*}
            y tomando ahora el límite  cuando $t$ tiende a $0$ por la derecha queda que
            \begin{gather*}
                0\leq -2(f-u,w-u) \Rightarrow (f-u,w-u) \leq 0
            \end{gather*}
        \end{itemize}
        Se deja como ejercicio demostrar la otra implicación y la unicidad de $u$.
    \end{proof}
\end{teo}

% 19 de septiembre
\begin{prop}
    La aplicación dada por
    \begin{align*}
        P_K: H &\to H\\
        f &\mapsto P_Kf
    \end{align*}
    es Lipschitziana, es decir, $\|P_Kf_1 - P_Kf_2\| \leq \|f_1-f_2\|$ para todo $f_1,f_2\in H$.

    \begin{proof}
        Tomamos $f_1,f_2\in H$ y consideramos $u_1 = P_Kf_1$, $u_2 = P_Kf_2$ y tenemos que 
        \begin{gather*}
            (f_1-u_1, v-u_1) \leq 0 \ \ \ \forall v \in K\\
            (f_2-u_2, v-u_2) \leq 0 \ \ \ \forall v \in K
        \end{gather*}
        De aquí obtenemos que 
        \begin{gather*}
            (f_1-u_1, u_2-u_1) \leq 0\\
            (f_2-u_2, u_1-u_2) \leq 0
        \end{gather*}
        Aprovechando la bilinealidad tenemos que
        \begin{gather*}
            (f_2-u_2, u_2-u_1) \geq 0 \Rightarrow ((f_1-u_1)-(f_2-u_2), u_2-u_1) \leq 0
        \end{gather*}
        Y además
        \begin{gather*}
            ((f_1-u_1)-(f_2-u_2), u_2-u_1) = ((f_1-f_2)-(u_1-u_2), u_2-u_1) =\\= (f_1-f_2, u_2-u_1) + (u_2-u_1, u_2-u_1)
        \end{gather*}
        Y aplicando la desigualdad de Cauchy-Schwarz
        \begin{align*}
            \|u_2-u_1\|^2 &= (u_2-u_1, u_2-u_1) \leq -(f_1-f_2, u_2-u_1)\\
            &\leq \|f_1-f_2\| \|u_2-u_1\| \Rightarrow \|u_2-u_1\| \leq \|f_1-f_2\|
        \end{align*}
    \end{proof}
\end{prop}

\begin{coro}[Proyección ortogonal]
    Sea $H$ un espacio de Hilbert y $\emptyset \neq M \subset H$ un subespacio vectorial cerrado. Entonces se tiene que 
    \begin{gather*}
        \forall f \in H \ \ \ \exists_1u\in M \text{ tal que } \|f-u\| = dist(f,M)
    \end{gather*}
    Además, $u$ está caracterizado por
    \begin{itemize}
        \item $u\in M$
        \item $(f-u, w) = 0$ \ \ \ $\forall w \in M$
    \end{itemize}

    Además, $P_M:H \to H$ es lineal.
    \begin{proof}
        Comencemos con la primera parte del corolario. Sabemos que $u\in K$ y $(f-u, v-u)\leq 0$\ \ \ $\forall v \in M$ del teorema de la proyección. Tendremos que probar la equivalencia entre esto y $(f-u, w) = 0$ \ \ \ $\forall w \in M$ cuando $M$ es un subespacio vectorial. Veamos ambas implicaciones:
        \begin{itemize}
            \item[$\Leftarrow$)] Evidente por ser $M$ un espacio vectorial.
            \item[$\Rightarrow$)] Tenemos que $(f-u, v-u)\leq 0$\ \ \ $\forall v \in M$. Tomamos ahora $v\in M$, $t\neq 0$ y como $M$ es un subespacio vectorial, entonces $\frac{v}{t}\in M$ por lo que
            
            \begin{gather*}
                (f-u, \frac{v}{t}-u) \leq 0 \ \ \ \forall v\in M, \ t\neq 0
            \end{gather*}
            Hagamos una distinción de casos:
            \begin{gather*}
                \left\{
                    \begin{array}{c c c}
                        \text{Si } t>0 & \Rightarrow  &(f-u,v-tu) \leq 0 \ \ \ \forall t>0, v\in M\\
                        \text{Si } t<0 & \Rightarrow  &(f-u,v-tu) \geq 0 \ \ \ \forall t<0, v\in M\\
                    \end{array}
                \right.
            \end{gather*}
            Tomando límite cuando $t$ tiende a $0$
            \begin{gather*}
                \left\{
                    \begin{array}{c}
                        (f-u,v) \leq 0 \ \ \ \forall t>0, v\in M\\
                        (f-u,v) \geq 0 \ \ \ \forall t<0, v\in M\\
                    \end{array}
                \right.
            \end{gather*}
            Y por tanto $(f-u,v) = 0$ \ \ \ $\forall v\in M$
        \end{itemize}

        La demostración de que $P_M$ es lineal se deja como ejercicio.
    \end{proof}
\end{coro}

\section{Espacios Duales}

\begin{definicion}[Dual algebráico]
    Sea $E$ un espacio vectorial, llamamos \textbf{dual algebráico} al siguiente espacio:
    \begin{gather*}
        E^\# = \{f: E \to \bb{R}: f \text{ es lineal}\}
    \end{gather*}
\end{definicion}

\begin{definicion}[Dual topológico]
   Dado $(E, \|\cdot\|)$ un espacio normado, llamamos \textbf{dual topológico} a
    \begin{gather*}
        E^\# = \{f: E \to \bb{R}: f \text{ es lineal y continua}\}
    \end{gather*} 
\end{definicion}


\begin{observacion}
    Si tenemos $(E, \|\cdot\|_E)$, $(F, \|\cdot\|_F)$ dos espacios normados y una aplicación $T:E\to F$ lineal. Son equivalentes:
    \begin{enumerate}
        \item[(i)] $T$ es continua
        \item[(ii)] $T$ es continua en $0$
        \item[(iii)] $T(B_E(0,1))$ es un conjunto acotado de $F$, es decir que $\exists R>0 : \|T(x)\|_F \leq R\ \ \ \forall x\in E$ con $\|x\|<1$
        \item[(iv)] $T$ es acotada, es decir, $T(A)$ es acotada en $F$ para todo $A\subset E$ que esé acotado
        \item[(v)] $T$ es Lipschitziana.
    \end{enumerate}
    La demostración se deja como ejercicio.
\end{observacion}

\begin{definicion}
    Dado $E$ un espacio vectorial, consideramos su dual topológico $E^*$ y definimos la norma
    \begin{gather*}
        \|f\|_{E^*} := \sup_{\|x\|\leq 1} \|f(x)\| \ \ \ \forall f \in E^*
    \end{gather*}
\end{definicion}

\begin{ejercicio} % Hacer para el miércoles que viene
    Demostrar que $\|f\|_{E^*}$ es una norma.
\end{ejercicio}

\begin{ejercicio}
    Demostrar que $(E^*, \|\cdot\|_{E*})$ es de Banach.
\end{ejercicio}

\begin{ejercicio}
    Demostrar que $\|f\|_{E^*} = \inf\{M\geq 0 : \|f(x)\| \leq M\|x\|_E\ \ \forall x \in E\}$
\end{ejercicio}

\section{Espacio Dual de un Espacio de Hilbert}

\begin{observacion}
    Es elemental que si tomo $v\in H$, entonces la aplicación
    \begin{align*}
        \varphi_v : H &\to \bb{R}\\
        u &\mapsto \varphi(u) = (u,v)
    \end{align*}
    verifica que $\varphi_v\in H^*$ y $\|\varphi_v\|_{H^*} = \|v\|_H$. Además, podemos definir la siguiente aplicación:
    \begin{align*}
        \Psi: H &\to H^*\\
        v &\mapsto \phi_v
    \end{align*}
    que será lineal por lo que tenemos que un espacio de Hilbert y su dual topológico serán isomorfos.

    \begin{proof}
        La demostración se deja como ejercicio.
    \end{proof}
\end{observacion}

\begin{teo}[Teorema de Riesz-Fischer]
    Para toda $\varphi\in H^*$, se tiene que $\exists_1 v\in H$ tal que $\varphi(u)=(u,v)\ \ \ \forall u\in H$. Además, se tiene que $\|\varphi\|_{H^*} = \|v\|_H$
\end{teo}
