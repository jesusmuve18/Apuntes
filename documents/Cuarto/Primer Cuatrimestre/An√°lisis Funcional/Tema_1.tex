\chapter*{Espacios de Hilbert}

Recordemos que un espacio de Hilbert es un par $(H, (\cdot, \cdot))$ donde $H$ es un espacio vectorial y $(\cdot, \cdot)$ es una función bilineal simétrica y definida positiva.

\begin{prop}
    Si $H$ es prehilbertiano entonces se tiene:
    \begin{enumerate}
        \item Se cumple la Desigualdad de Cauchy-Schwarz, es decir
        \begin{gather*}
            |(u,v)| \leq \|u\| \cdot \|v\|, \ \ \ \forall u,v\in H
        \end{gather*}
        \item Se verifica la desigualdad del paralelogramo
        \begin{gather*}
            \left\| \frac{u+v}{2} \right\|^2 + \left\| \frac{u-v}{2} \right\|^2 = \frac{1}{2}\left(\|u\|^2 + \|v\|^2\right), \ \ \ \forall u,v\in H
        \end{gather*}
    \end{enumerate}
\end{prop}

\begin{teo}[Teorema de la Proyección]
    Supongamos que $H$ es un espacio Hilbertiano y $\emptyset \neq K \subset H$ un conjunto convexo y cerrado, entonces $\forall f\in H$ $\exists_1 u\in K$ tal que $\|f-u\|=dist(f,K)$. Además, dicho $u$ está caracterizado por:
    \begin{gather*}
        \left\{
            \begin{array}{l}
                u\in K\\
                (f-u, v-u)\leq 0$ \ \ $\forall v \in K
            \end{array}
        \right.
    \end{gather*}
    Notaremos a dicha $u$ por $P_Kf$ y \textbf{diremos que es la proyección de $f$ sobre $K$}

    \begin{proof}
        En primer lugar tendremos que ver que $d(f,K)=\inf\{\|f-v\|: v\in K\}$ existe y se alcanza. Al ser un ínfimo de cantidades positivas sabemos que existe y nos quedará ver que se alcanza.

        Por definición de ínfimo tenemos que 
        \begin{gather*}
            \exists\{v_n\}\subset K \text{ tal que } \|f-v_n\|\to d
        \end{gather*}
        Aplicando la desigualdad del paralelogramo para $u=f-v_n$ y $v=f-v_m$, con $n,m\in \bb{N}$
        \begin{gather*}
            \left\| \frac{f-v_n + f-v_m}{2} \right\|^2 + \left\| \frac{f-v_n-(f-v_m)}{2} \right\|^2 = \frac{1}{2}\left(\|f-v_n\|^2 + \|f-v_m\|^2\right)\\
            \left\| f - \frac{v_n+v_m}{2} \right\|^2 + \left\| \frac{v_m - v_n}{2} \right\|^2 = \frac{1}{2}\left(\|f-v_n\|^2 + \|f-v_m\|^2\right)\\
            \frac{\left\| v_m-v_n \right\|^2}{4} = \frac{1}{2}\left(\|f-v_n\|^2 + \|f-v_m\|^2\right) - \left\| f - \frac{v_n+v_m}{2} \right\|^2\\
            \left\| v_m-v_n \right\|^2 = 2\left(\|f-v_n\|^2 + \|f-v_m\|^2\right) - 4\left\| f - \frac{v_n+v_m}{2} \right\|^2
        \end{gather*}
        Como $K$ es convexo y $v_n,v_m\in K$ tendremos que $d\frac{v_n+v_m}{2}\in K$ y además $\left\|f - \dfrac{v_n+v_m}{2}\right\|\geq d$ por lo que tenemos 
        \begin{gather*}
            \left\| v_m-v_n \right\|^2 = 2\left(\|f-v_n\|^2 + \|f-v_m\|^2\right) - 4d^2
        \end{gather*}
        Cuando $n\to \infty$ tenemos que $\|f-v_n\|\to d$ y $\|f-v_m\|\to d$ por lo que el término de la derecha tenderá a 0 cuando $n,m\to\infty$. Esto significa que la sucesión $\{v_n\}$ es de Cauchy.

        Como $H$ es de Hilbert, en particular es completo por lo que sabemos que $\{v_n\}\to u$ en $(H, (\cdot, \cdot))$.

        Como además $\{v_n\}\subset K$ y $K$ es cerrado, el límite $u\in K$. Tendremos que
        \begin{gather*}
            d = \lim_{n\to \infty} \|f-v_n\| = \|f-u\|
        \end{gather*}
        Y tendremos probada la existencia de $u$.\\

        Veamos ahora la equivalencia entre la primera y la segunda parte del teorema, es decir
        \begin{gather*}
            \left.
            \begin{array}{l}
                u\in K\\
                \|f-u\|=dist(f,K)
            \end{array}
            \right\} \sii
            \left\{
            \begin{array}{l}
                u\in K\\
                (f-u, v-u)\leq 0$ \ \ $\forall v \in K
            \end{array}
            \right.
        \end{gather*}
        Veamos las dos implicaciones:
        \begin{itemize}
            \item[$\Rightarrow$)] Supongamos que $u\in K$ y sabemos que $\|f-u\|\leq \|f-v\|$ para todo $v\in K$. Tomamos ahora $w\in K$ y consideramos el segmento que une $u$ con $w$. Entonces $\forall w\in K$ y $\forall t \in [0,1]$, al ser $K$ convexo tendremos que
            \begin{gather*}
                (1-t)u + tw \in K\ \  \text{ y }\ \ \|f-u\|^2 \leq \|f-(1-t)u-tw\|^2
            \end{gather*}
            Aplicando la bilinealidad podemos reescribir esta última expresión como 
            \begin{align*}
                \|f-(1-t)u-tw\|^2 &= (f-(1-t)u-tw,f-(1-t)u-tw) =\\
                &=\|f-u\|^2 + t^2\|w-u\|^2-2t(f-u,w-u)
            \end{align*}
            Sustituyendo en la expresión que teníamos anteriormente nos queda que:
            \begin{gather*}
                0\leq t^2\|w-u\|^2-2t(f-u,w-u) \ \ \ \forall t \in (0,1]
            \end{gather*}
            Al dividir entre $t$ nos queda
            \begin{gather*}
                0\leq t\|w-u\|^2-2(f-u,w-u) \ \ \ \forall t \in (0,1]
            \end{gather*}
            y tomando ahora el límite  cuando $t$ tiende a $0$ por la derecha queda que
            \begin{gather*}
                0\leq -2(f-u,w-u) \Rightarrow (f-u,w-u) \leq 0
            \end{gather*}
        \end{itemize}
        Se deja como ejercicio demostrar la otra implicación y la unicidad de $u$.
    \end{proof}
\end{teo}
