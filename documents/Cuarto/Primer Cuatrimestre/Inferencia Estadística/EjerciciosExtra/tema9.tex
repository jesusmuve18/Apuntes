\documentclass[12pt]{book}

\input{../../../../_assets/preambulo.tex}
\usepackage{xkeyval} % Para el paso de argumentos

\usepackage{graphicx}

% Definir la carpeta de las imágenes
\graphicspath{{../_assets}{../../_assets}}

% Definir el comando \portada
\makeatletter
\define@key{portada}{titulo}{\def\titulo{#1}}
\define@key{portada}{subtitulo}{\def\subtitulo{#1}}
\define@key{portada}{autor}{\def\autor{#1}}
\define@key{portada}{año}{\def\año{#1}}

\newcommand*{\portada}[1][]{%
    % Definimos las claves y sus valores por defecto
  \setkeys{portada}{%
    titulo=Sin Título,%
    subtitulo=Sin Subtítulo,%
    autor=Autor Desconocido,%
    año=Sin Año, #1}%
    \begin{titlepage}
        \centering
        {\includegraphics[width=0.2\textwidth]{Logo-UGR-Black.png}\par}
        \vspace{1cm}
        {\bfseries\LARGE Universidad de Granada \par}
        \vspace{1cm}
        {\scshape\Large Doble Grado en Ingeniería Informática y Matemáticas \par}
        \vspace{3cm}
        {\scshape\Huge \titulo \par}
        \vspace{3cm}
        {\itshape\Large \subtitulo \par}
        \vfill
        {\Large Autor: \par}
        {\Large \autor \par}
        \vfill
        {\Large \año \par}
    \end{titlepage}%
}

\begin{document}
    \portada[%
        titulo=Inferencia Estadística,
        subtitulo=Tema 9. Participación extra,
        autor=Jesús Muñoz Velasco,
        año=Curso 2025-2026]
        
    \begin{ejercicio}
        En un estudio sobre el funcionamiento de un servidor web, se registran los accesos recibidos durante una hora por un sistema de balanceo de carga que distribuye las peticiones entre 8 procesos idénticos. Durante el periodo de observación se contabilizan 160 peticiones, obteniéndose el siguiente número de accesos gestionados por cada proceso:
        \begin{gather*}
            18, 21, 19, 17, 23, 20, 22, 20
        \end{gather*}
        Suponiendo que, si el sistema funciona correctamente, cada petición tiene la misma probabilidad de ser asignada a cualquiera de los procesos, contrastar si los datos son compatibles con una distribución uniforme de las peticiones entre los procesos.

        \hrulefill

        \section*{Resolución}

        El ejercicio consiste en contrastar si la asignación de peticiones entre los procesos del servidor se realiza de forma uniforme. Para verlo definimos la variable aleatoria:
        \begin{gather*}
            X \equiv \text{proceso que gestiona una posición}
        \end{gather*}
        La variable $X$ toma valores en un conjunto de 8 categorías, correspondientes a los procesos del sistema.\\

        Buscamos ahora plantear el contraste a analizar. Bajo un funcionamiento correcto del balanceador de carga, cada proceso debería recibir la misma proporción de peticiones. Por tanto, el contraste es:
        \begin{gather*}
            \left\{
                \begin{array}{l}
                    H_0:P(X=i)=\frac{1}{8}\ \text{ para todo } i\in \{1,...,8\}\\
                    H_1:P(X=i)\neq \frac{1}{8}\ \text{ para algún } i\in \{1,...,8\}
                \end{array}
            \right.
        \end{gather*}
        e trata de un contraste de bondad de ajuste, que resolvemos mediante el test $\chi^2$.\\

        Bajo la hipótesis de distribución uniforme para $n=160$ peticiones, el número esperado de peticiones en cada proceso es $np_i^0=\nicefrac{160}{8} = 20\  (\geq 5)$, $i\in\{1,...,8\}$ y por tanto, el estadístico $\chi^2$ toma el valor
        \begin{align*}
            \chi^2_{exp} &= \sum\limits_{i=1}^8 \frac{(n_i - np_i^0)^2}{np_i^0} = -n + \frac{1}{np_i^0} \sum\limits_{i=1}^8n_i^2 =\\
            &= -160 + \frac{1}{20} (18^2 + 21^2 + 19^2 + 17^2 + 23^2 + 20^2 + 22^2 + 20^2) =\\
            &= 1.4
        \end{align*}
        Como la distribución de $\chi^2(N_1,...,N_8)$ bajo $H_0$ es $\chi^2(7)$ tendremos que el $p$-nivel asociado es 
        \begin{gather*}
            p-\text{nivel}:\ \ P_{H_0}(\chi^2(N_1,...,N_8)>1.4) \in (0.975, 0.99)
        \end{gather*}
        Dado que el $p$-nivel es muy elevado, no se rechaza la hipótesis nula a ningún nivel de significación razonable.\\

        Por tanto, los datos son compatibles con una distribución uniforme de las peticiones entre los procesos, y no se detectan indicios de un mal funcionamiento del sistema de balanceo de carga.

    \end{ejercicio}
\end{document}