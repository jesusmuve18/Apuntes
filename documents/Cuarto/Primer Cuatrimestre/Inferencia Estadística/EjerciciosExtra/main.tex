\documentclass[12pt]{book}

\input{../../../../_assets/preambulo.tex}
\usepackage{xkeyval} % Para el paso de argumentos

\usepackage{graphicx}

% Definir la carpeta de las imágenes
\graphicspath{{../_assets}{../../_assets}}

% Definir el comando \portada
\makeatletter
\define@key{portada}{titulo}{\def\titulo{#1}}
\define@key{portada}{subtitulo}{\def\subtitulo{#1}}
\define@key{portada}{autor}{\def\autor{#1}}
\define@key{portada}{año}{\def\año{#1}}

\newcommand*{\portada}[1][]{%
    % Definimos las claves y sus valores por defecto
  \setkeys{portada}{%
    titulo=Sin Título,%
    subtitulo=Sin Subtítulo,%
    autor=Autor Desconocido,%
    año=Sin Año, #1}%
    \begin{titlepage}
        \centering
        {\includegraphics[width=0.2\textwidth]{Logo-UGR-Black.png}\par}
        \vspace{1cm}
        {\bfseries\LARGE Universidad de Granada \par}
        \vspace{1cm}
        {\scshape\Large Doble Grado en Ingeniería Informática y Matemáticas \par}
        \vspace{3cm}
        {\scshape\Huge \titulo \par}
        \vspace{3cm}
        {\itshape\Large \subtitulo \par}
        \vfill
        {\Large Autor: \par}
        {\Large \autor \par}
        \vfill
        {\Large \año \par}
    \end{titlepage}%
}

\begin{document}
    \portada[%
        titulo=Inferencia Estadística,
        subtitulo=Tema 8. Participación extra,
        autor=Jesús Muñoz Velasco,
        año=Curso 2025-2026]
        
    \begin{ejercicio}
        El departamento de Recursos Humanos de una empresa de logística quiere analizar si existe una relación lineal entre la antigüedad de los empleados (en años) y su productividad mensual (medida en miles de unidades procesadas). Para ello, selecciona una muestra aleatoria de 8 empleados y obtiene los siguientes datos:
        \begin{center}
            \begin{tabular}{|l | c c c c c c c c|}
                \hline
                Antigüedad (años) & 2 & 4 & 6 & 8 & 10 & 12 & 14 & 16 \\
                Productividad (miles uds./mes) & 14 & 16 & 26 & 38 & 30 & 42 & 54 & 55\\
                \hline
            \end{tabular}
        \end{center}
        \begin{enumerate}
            \item[a)] Obtener la recta de regresión estimada de la productividad sobre la antigüedad e interpretar sus coeficientes.
            \item[b)] Descomponer la variabilidad (VT, VE, VNE), obtener la varianza residual ($S_R^2$) y calcular el coeficiente de determinación ($r^2$), interpretando el resultado.
            \item[c)] Predecir la productividad para un empleado con 11 años de antigüedad.
            \item[d)] Suponiendo las hipótesis de normalidad, contrastar al nivel de significación $\alpha=0.05$ si existe una relación lineal significativa entre ambas variables.
        \end{enumerate}
    \end{ejercicio}
    \hrulefill
    \section*{Resolución}
    \begin{enumerate}
        \item[a)] De los datos aportados podemos construir la siguiente tabla
        \begin{center}
            \begin{tabular}{c | c c c c c}
                & $x_i$ & $y_i$ & $x_i^2$ & $x_iy_i$ & $y_i^2$\\
                \hline
                & 2 & 14 & 4 & 28 & 196\\
                & 4 & 16 & 16 & 64 & 256 \\
                & 6 & 26 & 36 & 156 & 676 \\
                & 8 & 38 & 64 & 304 & 1444 \\
                & 10 & 30 & 100 & 300 & 900 \\
                & 12 & 42 & 144 & 504 & 1764 \\
                & 14 & 54 & 196 & 756 & 2916 \\
                & 16 & 55 & 256 & 880 & 3025\\
                \hline
                Sumas & 72 & 275 & 816 & 2992 & 11177\\
            \end{tabular}
        \end{center}
        Pasamos a calcular con estos datos las medidas necesarias para el cálculo de la recta de regresión:
        \begin{gather*}
            \overline{x} = \frac{72}{8} = 9 \text{ años}\\
            \overline{y} = \frac{275}{8} = 34.375 \text{ miles uds./mes}\\
            \sigma_x^2 = \frac{816}{8} -9^2 = 102 -81 = 21\\
            \sigma_y^2 = \frac{11177}{8} - 34,375^2 \cong  1397,125 - 1181.64 = 215,485\\
            \sigma_{xy} = \frac{2992}{8} - (9 \cdot 34,375) = 374 - 309,375 = 64,625
        \end{gather*}
        Como sabemos que la fórmula de la recta de regresión viene dada por 
        \begin{gather*}
            y = \bar{y} + \frac{\sigma_{xy}}{\sigma_x^2}(x - \bar{x})
        \end{gather*}
        podemos sustituir los valores recién calculados obteniendo finalmente la recta de regresión.
        \begin{gather*}
            y = 34,375 + \frac{64,625}{21}(x-9) \cong 6,6786 + 3,0774x
        \end{gather*}
        Podemos interpretar este resultado de la siguiente forma:
        \begin{itemize}
            \item Por cada año adicional de antigüedad, la productividad aumenta en promedio unas $3,0774$ miles de unidades.
            \item Un empleado recién contratado ($x=0$) tendría una productividad base estimada de $6,6786$ miles de unidades mensuales.
        \end{itemize}

        \item[b)] Comenzamos calculando la variabilidad:
        \begin{gather*}
            VT = n\sigma_y^2 = 8 \cdot 215,485 = 1723,88\\
            VE = \frac{n\sigma_{xy}^2}{\sigma_x^2} = \frac{8 \cdot 64,625^2}{21} \cong 1591,006\\
            VNE = VT -VE = 1723,88 - 1591,006 = 132,874
        \end{gather*}
        Calculamos ahora la varianza residual:
        \begin{gather*}
            S_R^2 = \frac{VNE}{n-2} = \frac{132,874}{6} \cong 22,1457
        \end{gather*}
        Calculamos el coeficiente de determinación:
        \begin{gather*}
            r^2 = \frac{VE}{VT} = \frac{1591,006}{1723,88} \cong 0,9229
        \end{gather*}
        A partir de este resultado podemos concluir que el modelo estimado explica el $92,29\%$ de la variabilidad de la productividad. El ajuste es por tanto muy bueno por lo que hay una relación lineal bastante considerable. 

        \item[c)] Calculamos ahora la productividad para un empleado con $x_p=11$ años de antigüedad a partir de la recta de regresión calculada anteriormente:
        \begin{gather*}
            \hat{y}_p = 6,6786 + 3,0774x_p = 6,6786 + 3,0774\cdot 11 = 40,53 \text{ miles uds./mes}
        \end{gather*}
        
        \item[d)] Planteamos el contraste para confirmar si existe relación lineal
        \begin{gather*}
            \left\{
                \begin{array}{c}
                    H_0: \beta_1 = 0 \text{ (No hay relación lineal)}\\
                    H_1: \beta_1 \neq 0 \text{ (Existe relación lineal)}
                \end{array}
            \right.
        \end{gather*}
        donde para $n=7$ y $\alpha = 0,05$
        \begin{gather*}
            \varphi(Y) = \left\{
                \begin{array}{ccc}
                    1 & \text{ si } & F(Y) > F_{1,n-2;\alpha} \\
                    0 & \text{ si } & F(Y) \leq F_{1,n-2;\alpha} 
                \end{array}
            \right.
        \end{gather*}
        Consideramos entonces el siguiente estadístico de prueba
        \begin{gather*}
            F_{exp} =  \frac{VE}{S_R^2} = \frac{1591,006}{22,1457} \cong 71,8426
        \end{gather*}
        de la tabla $F$ de Snedecor tenemos que $F_{1, 6; 0.05} \cong 5.99$. Como $F_{exp} > F_{1,6;0,05}$ rechazamos la hipótesis $H_0$, es decir, los datos aportan evidencia de una relación lineal. Esto concuerda con el valor tan alto que hemos obtenido al calcular $r^2$. Finalmente podemos afirmar que existe evidencia estadística significativa, al $95\%$ de confianza, para concluir que la productividad depende linealmente de la antigüedad.
    \end{enumerate}
\end{document}