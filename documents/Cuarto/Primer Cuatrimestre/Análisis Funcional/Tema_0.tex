\chapter*{Repaso}

\begin{definicion}[Espacio normado]
    $E$ un espacio vectorial y $\|\cdot\|: E \to \bb{R}$ una función que verifica:
    \begin{enumerate}
        \item $\|x\| \geq 0$ $\forall x \in E$
        \item $\|x\| = 0 \sii x=0$
        \item $\|x + y\| \leq \|x\| + \|y\|$
        \item $\|\lambda x\| = |\lambda|$ $\forall x,y\in E$, $\lambda \in \bb{R}$
    \end{enumerate} 

    A esta función la llamaremos \textbf{norma} y diremos que $E$ es un \textbf{espacio normado}

    Podemos definir además una función $d:E\times E \to \bb{R}$ dada por $d(x,y) = \|x-y\|$ $\forall x,y \in E$ a la que llamaremos \textbf{distancia}.

    Decimos que un espacio $E$ es \textbf{completo} si toda sucesión de Cauchy es convergente.

    Si $E$ es un espacio normado completo, entonces $(E, \|.\|)$ es un \textbf{espacio de Banach}.
\end{definicion}

\begin{definicion}[Espacio prehilbertiano]
    Sea $H$ es un espacio vectorial, un \textbf{producto escalar} es una función $(\cdot,\cdot):H\times H \to \bb{R}$ tal que verifica las siguientes propiedades:
    \begin{enumerate}
        \item \textbf{Bilineal:} para todo $x,y,z\in H$, $\alpha, \beta \in \bb{R}$ se verifica que 
        \begin{align*}
            (\alpha x + \beta y, z) = \alpha(x,z) + \beta(y,z)\\
            (x, \alpha y + \beta z) = \alpha(x,y) + \beta(x,z)
        \end{align*}

        \item \textbf{Simétrica:} $(x,y) = (y,x) \ \ \ \forall x,y\in H$
         
        \item \textbf{Positiva:} $(x,x)\geq 0  \ \ \ \forall x \in H$

        \item \textbf{Definida positiva:} $(x,x) > 0 \ \ \ \forall x \in H\setminus\{0\}$
    \end{enumerate}

    Las dos últimas propiedades se pueden resumir en $(x,x) = 0 \sii x=0$.
    
    Diremos que $(H, (\cdot, \cdot))$ es un \textbf{espacio prehilbertiano}.

    Todo espacio prehilbertiano es en particular un espacio normado, ya que podemos definir $\|x\| = \sqrt{(x,x)}$ que es claramente una norma.\\

    Si $\|\cdot\|$ es completa, diremos que $(H, (\cdot,\cdot))$ es un \textbf{espacio de Hilbert}.
\end{definicion}

\begin{ejemplo}Los siguientes espacios son de Banach:\\
    \begin{enumerate}
        \item $(\bb{R}, |\cdot|)$.
        \item $(\bb{R}^N, |\cdot|)$, donde $|x| = |(x_1, x_2, \dots, x_N)| = \sqrt{x_1^2 + x_2^2 + \dots + x_N^2}$. Además es de Hilbert ya que $(x,y) = \sum_{i=1}^N x_i y_i$ es un producto escalar.
        \item  Dado\footnote{la $b$ de $\cc{C}_b$ viene de \textit{bounded} (acotado en inglés)} $A\subset \bb{R}^N$ tomamos $\cc{C}_b(A) = \{f: A \to \bb{R} : f \text{ es continua y acotada en } A \}$. Podemos definir una norma en este espacio como 
        \begin{align*}
            \|f\|_{\cc{C}_b(A)} = \sup\{|f(x)| : x\in A\}
        \end{align*}

        \item Tomamos $K\subset \bb{R}^N$ compacto. Consideramos el conjunto de las funciones continuas en $K$ denotado por $\cc{C}(K)$ y el espacio $(K, (\cdot,\cdot))$, donde 
        \begin{gather*}
            (f,g) = \int_K f(x)g(x) dx
        \end{gather*}
        es un producto escalar que hace a este un espacio prehilbertiano. Tendríamos 
        \begin{gather*}
            \|f\| = \left(\displaystyle\int_K f(x)^2 dx\right)^{1/2}
        \end{gather*}
    \end{enumerate}
\end{ejemplo}

\begin{ejemplo}[El espacio del punto 4 No es de Hilbert]

    Veámoslo con un contraejemplo.

    Tomamos $K = [0,1]\subset \bb{R}$ y podemos definir $\forall n \in \bb{N}$ la función  $f_n : [0,1] \to \bb{R}^+$ tal que $f_n^2$ viene dada por la siguiente gráfica:

    \begin{figure}[H]
        \centering
        \begin{tikzpicture}
            \begin{axis}[
                axis lines=left,
                xlabel={$x$},
                ylabel={$f_n^2(x)$},
                ylabel style={rotate=-90}, % asegura que esté vertical recto
                height=5cm, width=7cm,
                xmin=0, xmax=3,
                ymin=0, ymax=1.2,
                xtick={0,1,3},
                xticklabels={$0$,$\nicefrac{1}{n}$},
                ytick={1},
                yticklabels={$1$},
                grid=none
            ]
                % Línea descendente desde (0,1) hasta (1/n,0)
                \addplot[very thick,red!60!black] coordinates {(0,1) (1,0)};
                % Línea horizontal en 0 desde (1/n,0) hasta (3,0)
                \addplot[very thick,red!60!black] coordinates {(1,0.01) (3,0.01)};
            \end{axis}
        \end{tikzpicture}
    \end{figure}

    De esta forma tenemos que
    \begin{gather*}
        \|f_n\|^2 = \int_0^1 f_n^2(x) dx = \frac{1}{n} \cdot \frac{1}{2} = \frac{1}{2n} \Rightarrow \|f_n\| = \frac{1}{\sqrt{2n}} \to 0
    \end{gather*}

    y vemos que

    \begin{gather*}
        \left\{
        \begin{array}{c  c}
            \{f_n(x)\} \to 0 & \forall x\in (0,1]\\
            \{f_n(0) = 1\} \to 1 &\\
        \end{array}
        \right.
    \end{gather*}

    Con esto tenemos que la sucesión $\{f_n\} \to 0$ en $(\cc{L}([0,1]), (\cdot,\cdot))$ (ya que la norma converge a 0).\\

    PARA MAÑANA RESOLVER QUÉ ES LO QUE NO ESTÁ CLARO (la contradicción para ser espacio de Hilbert).\\
\end{ejemplo}

\begin{ejemplo}
    Consideramos $\emptyset \neq \Omega \subset \bb{R}^N$ medible, entonces podemos definir
    \begin{gather*}
        L^2(\Omega) = \cc{L}^2(\Omega)/\sim = \{f:\Omega \to \bb{R} \text{ medible } : \int_\Omega f(x)^2 dx < \infty\}
    \end{gather*}
    $L^2(\Omega)$ con la norma definida anteriormente (en el punto 4) es un espacio de Hilbert (teorema de Fischer)
\end{ejemplo}

\begin{ejemplo}
    Sea $1 \leq p < \infty$. Consideramos el conjunto
    \begin{gather*}
        L^p(\Omega) = \left\{f: \Omega \to \bb{R} \text{ medibles } : \int_\Omega |f|^p dx < \infty\right\}
    \end{gather*}
    Entonces tenemos que con la norma definida como 
    \begin{gather*}
        \|f\|_{L^p(\Omega)} = \left(\int_\Omega |f|^p dx \right)^{1/p}
    \end{gather*}
    es un espacio de Banach. Recordemos para este resultado la desigualdad de Hölder y Minkowski. Definimos para ello el conjugado de $p$ de la siguiente forma\footnote{donde asumimos que $\nicefrac{1}{\infty} = 0$}:
    \begin{gather*}
        p' = \left\{
        \begin{array}{c c l}
            \frac{p}{p-1} & \text{ si } & 1<p<\infty\\\\
            \infty & \text{ si } & p=1
        \end{array}
        \right\} \Rightarrow \frac{1}{p} + \frac{1}{p'} = 1 \ \ \ \forall p \in [1, \infty)
    \end{gather*}
    Con esto tendremos que 
    \begin{gather*}
        \left.
            \begin{array}{c}
                f\in L^p(\Omega)\\
                g \in L^{p'}(\Omega)
            \end{array}
        \right\} \Rightarrow fg \in L^1(\Omega)
    \end{gather*}
    Además, se tiene que 
    \begin{gather*}
        \int |f(x)g(x)| dx \leq \left(\int |f|^p dx\right)^{\nicefrac{1}{p}} \left(\int |f|^{p'} dx\right)^{\nicefrac{1}{p'}} = \|f\|_{L^p} \|g\|_{L^{p'}}
    \end{gather*}
\end{ejemplo}

\begin{ejemplo}\ \\
    \begin{enumerate}
        \item $(\bb{R}^N, \|\cdot\|_ p)$ con $\|x\|_p = (\sum_{i=1}^N |x_i|^p)^{1/p}(x,y) = \sum_{i=1}^N x_i y_i$.
        \item $(\bb{R}^N, \|\cdot\|_\infty)$ con $\|x\|_\infty = \max \{|x_i|: i=1,\dots,N\}$
        \item Sea $p=\infty$. Tenemos
        \begin{gather*}
            L^{\infty} = \{f:\Omega \to \bb{R} \text{ medible } : \sup\{|f(x)| : x\in \Omega\}< \infty\}
        \end{gather*}
        A este supremo lo llamaremos \textbf{supremo esencial}, que se define de la siguiente forma\footnote{\textit{a.e} viene de \textit{almost everywhere} (casi por doquier en inglés)}:
        \begin{gather*}
            \sup_{\Omega}|f| = \inf \{M \geq 0 : |f(x)| \leq M \ \ a.e. \ x\in \Omega\}
        \end{gather*}        
        En algunos libros se denota por $ess\sup$.

        Podremos reescribir lo anterior como
        \begin{gather*}
            L^{\infty} = \{f:\Omega \to \bb{R} \text{ medible } : \sup_\Omega |f|< \infty\}
        \end{gather*}
        Entonces el espacio $(L^\infty, \|\cdot\|_\infty)$ con $\|f\|_\infty = \sup_\Omega |f|$ es un espacio de Banach.

        La desigualdad de Hölder con $p=\infty$, $p'=1$ nos dice que para $f \in L^\infty(\Omega)$, $g\in L^1(\Omega)$ entonces $fg \in L^1(\Omega)$ y $\|fg\|_{L^1}\leq \|f\|_{L^\infty} \|g\|_{L^{1}}$ es una norma en $H$.
    \end{enumerate}
\end{ejemplo}

\begin{ejemplo}
    Consideramos $1\leq p < \infty$ y definimos el conjunto de sucesiones.
    \begin{gather*}
        \cc{L}^p = \{ x : \bb{N} \to \bb{R} : \sum_{n=1}^{\infty}|x(n)|^p < \infty\}
    \end{gather*}
    Si definimos ahora
    \begin{gather*}
        \|x\|_{\cc{L}^p} = \left(\sum_{n=1}^{\infty}|x(n)|^p\right)^{\nicefrac{1}{p}}
    \end{gather*}
    entonces $(\cc{L}^p, \|\cdot\|_p)$ es un espacio de Banach. Para verlo podemos tomar $x\in \cc{L}^p$, $y\in \cc{L}^{p'}$ y tenemos que 
    \begin{gather*}
        xy\in \cc{L}^1 \text{\ \  y\ \  } \|xy\|_{\cc{L}^1} \leq \|x\|_{\cc{L}^p} \|y\|_{\cc{L}^{p'}}
    \end{gather*} de la que se deduce la desigualdad de Mikowsky.\\

    Para $p=2$ tenemos que $(\cc{L}^2, \|\cdot\|_2)$ es un espacio de Hilbert.     Para $p=\infty$ podemos definir $\cc{L}^{\infty} = \{x : \bb{N} \to \bb{R} : x \text{ sucesión acotada}\}$ y con $\|x\|_\infty = \sup\{|x(n)| : n\in \bb{N}\}$ es un espacio de Banach.
\end{ejemplo}

\begin{ejemplo} Podemos considerar los siguientes subespacios que seguirán siendo espacios de Banach:
    \begin{enumerate}
        \item Tomamos $C = \{x\in \cc{L}^{\infty} : x \text{ es convergente}\}$ y es un subespacio de $\cc{L}^\infty$. 
        \item Podemos tomar otro subespacio de este, $C_0 = \{x\in C : x \text{ es convergente a } 0\}$ que de nuevo es un subespacio de $\cc{L}^{\infty}$.
    \end{enumerate}    
\end{ejemplo}



