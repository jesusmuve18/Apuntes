\chapter*{Introducción}

Comenzaremos la introducción al contenido de esta asignatura recordando brevemente el concepto de cuerpo\footnote{\textit{field} en inglés}. Lo primero que sabemos es que un cuerpo es un tipo de anillo conmutativo. Un anillo\footnote{\textit{ring} en inglés} es un conjunto no vacío, $A$ que tiene definidas dos aplicaciones binarias y dos elementos especiales, $(A, +, 0, \cdot, 1)$. Con $(+,0)$ tenemos que $A$ es un grupo aditivo y con $(\cdot, 1)$ tenemos que $A$ es un monoide, es decir, que cuenta con una aplicación asociativa con elemento neutro $1$. Además estas 2 operaciones tienen que guardar una cierta compatibilidad (axiomas), que llamamos leyes distributivas y que son los siguientes:
\begin{itemize}
    \item $a\cdot(b+c) = a\cdot b + a \cdot c$
    \item $(b+c)\cdot a = b\cdot a + c \cdot a$, \ \ $\forall a,b\in A$
\end{itemize}
Con esto habremos completado la definición de anillo. La conmutatividad hace referencia a la siguiente propiedad:
\begin{gather*}
    a\cdot b = b \cdot a  \ \ \forall a,b\in A
\end{gather*}
Veamos ahora qué tiene que suceder para que a este anillo conmutativo lo llamemos cuerpo. Para ello, es equivalente decir que $A\setminus \{0\}$ es un grupo y que $\forall a \in A\setminus\{0\}$ existe un $a^{-1}\in A\setminus \{0\}$ tal que $a\cdot a^{-1} = 1$ (lo cual implica claramente $0\neq 1$).

\begin{ejemplo}\
    \begin{itemize}
        \item Los racionales, $\bb{Q}$.
        \item Los reales, $\bb{R}$.
        \item Los complejos, $\cc{C}$.
        \item $\bb{Z}_p$ con $p$ primo.
    \end{itemize}
\end{ejemplo}

\begin{notacion}
    Denotaremos el producto de 2 elementos por yuxtaposición\footnote{Las matemáticas son el arte de ser ambiguo siendo preciso en cada instante (Torrecillas, 18-9-2025)}, es decir, $a\cdot b = ab$
\end{notacion}

Recordaremos ahora los conceptos de subanillo y subcuerpo. Para ello consideramos $A$ un anillo y un subconjunto $B\subseteq A$ tal que $1\in B$. Si además tenemos que $(B,+)$ es un subgrupo de $(A, +)$ y que para todo $a,b\in B$ se tiene que $ab\in B$, entonces diremos que $B$ es un subanillo de $A$.

\begin{ejemplo}\
    \begin{itemize}
        \item $\bb{Z}$ es subanillo de $\bb{Q}$.
        \item $\bb{Q}$ es subanillo de $\bb{R}$.
        \item $\bb{R}$ es subanillo de $\bb{C}$
    \end{itemize}
\end{ejemplo}

\begin{definicion}[Homomorfismo de anillos]
    Dados $A$ y $B$ dos anillos, un \textbf{homomorfismo} $f:A\to B$ es una aplicación que verifica para todo $a,b\in A$ las siguientes propiedades:
    \begin{itemize}
        \item $f(1)=1$
        \item $f(a+b) = f(a) + f(b)$
        \item $f(ab) = f(a)f(b)$
    \end{itemize}
\end{definicion}

\begin{definicion}[Característica de un anillo]
    Dado $A$ un anillo, existe un único homomorfismo de anillos\footnote{se prueba fácilmente por inducción} $\chi:\bb{Z} \to A$. Entonces $\ker\chi$ es un ideal de $\bb{Z}$ y por tanto será principal, es decir, que $\ker\chi = n\bb{Z}$ para cierto $n\in \bb{N}$. Dicho $n$ es el número al que llamaremos \textbf{característica} de $A$ y la notaremos como $n=car(A)$.
\end{definicion}

\begin{definicion}[Subanillo]
    Si $K$ es un cuerpo, entonces un subcuerpo de $K$ es un \textbf{subanillo} $F$ de $K$ tal que $F$ es un cuerpo.
\end{definicion}

\begin{observacion}
    Sea $K$ un cuerpo y $\Gamma$ un conjunto no vacío\footnote{El propio $K$ está en este conjunto} de subcuerpos de $K$. Entonces $\bigcap\limits_{F\in\Gamma}F$ es un subcuerpo de $K$.
\end{observacion}

\begin{definicion}[Subcuerpo primo]
    Sea $K$ un cuerpo y tomamos $S\subset K$ un subconjunto y consideramos
    \begin{gather*}
        \Gamma = \{\text{ subcuerpos de }K \text{ que contienen a }S\}
    \end{gather*}
    En $\Gamma$ podemos tomar la intersección, $\bigcap\limits_{F\in\Gamma}F$ que es el subgrupo más pequeño que contiene a $S$. Para $S=\emptyset$ obtengo el menor subcuerpo de $K$ y a este subcuerpo lo llamaremos \textbf{subcuerpo primo} de $K$.
\end{definicion}

