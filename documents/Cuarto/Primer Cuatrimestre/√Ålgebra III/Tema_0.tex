\chapter*{Introducción}

Comenzaremos la introducción al contenido de esta asignatura recordando brevemente el concepto de cuerpo\footnote{\textit{field} en inglés}. Lo primero que sabemos es que un cuerpo es un tipo de anillo conmutativo. Un anillo\footnote{\textit{ring} en inglés} es un conjunto no vacío, $A$ que tiene definidas dos aplicaciones binarias y dos elementos especiales, $(A, +, 0, \cdot, 1)$. Con $(+,0)$ tenemos que $A$ es un grupo aditivo y con $(\cdot, 1)$ tenemos que $A$ es un monoide, es decir, que cuenta con una aplicación asociativa con elemento neutro $1$. Además estas 2 operaciones tienen que guardar una cierta compatibilidad (axiomas), que llamamos leyes distributivas y que son los siguientes:
\begin{itemize}
    \item $a\cdot(b+c) = a\cdot b + a \cdot c$
    \item $(b+c)\cdot a = b\cdot a + c \cdot a$, \ \ $\forall a,b\in A$
\end{itemize}
Con esto habremos completado la definición de anillo. La conmutatividad hace referencia a la siguiente propiedad:
\begin{gather*}
    a\cdot b = b \cdot a  \ \ \forall a,b\in A
\end{gather*}
Veamos ahora qué tiene que suceder para que a este anillo conmutativo lo llamemos cuerpo. Para ello, es equivalente decir que $A\setminus \{0\}$ es un grupo y que $\forall a \in A\setminus\{0\}$ existe un $a^{-1}\in A\setminus \{0\}$ tal que $a\cdot a^{-1} = 1$ (lo cual implica claramente $0\neq 1$).

\begin{ejemplo}\
    \begin{itemize}
        \item Los racionales, $\bb{Q}$.
        \item Los reales, $\bb{R}$.
        \item Los complejos, $\cc{C}$.
        \item $\bb{Z}_p$ con $p$ primo.
    \end{itemize}
\end{ejemplo}

\begin{notacion}
    Denotaremos el producto de 2 elementos por yuxtaposición\footnote{Las matemáticas son el arte de ser ambiguo siendo preciso en cada instante (Torrecillas, 18-9-2025)}, es decir, $a\cdot b = ab$
\end{notacion}

Recordaremos ahora los conceptos de subanillo y subcuerpo. Para ello consideramos $A$ un anillo y un subconjunto $B\subseteq A$ tal que $1\in B$. Si además tenemos que $(B,+)$ es un subgrupo de $(A, +)$ y que para todo $a,b\in B$ se tiene que $ab\in B$, entonces diremos que $B$ es un subanillo de $A$.

\begin{ejemplo}\
    \begin{itemize}
        \item $\bb{Z}$ es subanillo de $\bb{Q}$.
        \item $\bb{Q}$ es subanillo de $\bb{R}$.
        \item $\bb{R}$ es subanillo de $\bb{C}$
    \end{itemize}
\end{ejemplo}

\begin{definicion}[Homomorfismo de anillos]
    Dados $A$ y $B$ dos anillos, un \textbf{homomorfismo} $f:A\to B$ es una aplicación que verifica para todo $a,b\in A$ las siguientes propiedades:
    \begin{itemize}
        \item $f(1)=1$
        \item $f(a+b) = f(a) + f(b)$
        \item $f(ab) = f(a)f(b)$
    \end{itemize}
\end{definicion}

\begin{definicion}[Característica de un anillo]
    Dado $A$ un anillo, existe un único homomorfismo de anillos\footnote{se prueba fácilmente por inducción} $\chi:\bb{Z} \to A$. Entonces $\ker\chi$ es un ideal de $\bb{Z}$ y por tanto será principal, es decir, que $\ker\chi = n\bb{Z}$ para cierto $n\in \bb{N}$. Dicho $n$ es el número al que llamaremos \textbf{característica} de $A$ y la notaremos como $n=car(A)$.
\end{definicion}

\begin{definicion}[Subanillo]
    Si $K$ es un cuerpo, entonces un subcuerpo de $K$ es un \textbf{subanillo} $F$ de $K$ tal que $F$ es un cuerpo.
\end{definicion}

\begin{observacion}
    Sea $K$ un cuerpo y $\Gamma$ un conjunto no vacío\footnote{El propio $K$ está en este conjunto} de subcuerpos de $K$. Entonces $\bigcap\limits_{F\in\Gamma}F$ es un subcuerpo de $K$.
\end{observacion}

\begin{definicion}[Subcuerpo primo]
    Sea $K$ un cuerpo y tomamos $S\subset K$ un subconjunto y consideramos
    \begin{gather*}
        \Gamma = \{\text{ subcuerpos de }K \text{ que contienen a }S\}
    \end{gather*}
    En $\Gamma$ podemos tomar la intersección, $\bigcap\limits_{F\in\Gamma}F$ que es el subgrupo más pequeño que contiene a $S$. Para $S=\emptyset$ obtengo el menor subcuerpo de $K$ y a este subcuerpo lo llamaremos \textbf{subcuerpo primo} de $K$.
\end{definicion}

\begin{observacion}
    Si tenemos $\chi: \bb{Z} \to K$ el homomorfismo de anillos, de forma que $p$ es la característica de $K$, es decir, $p\bb{Z} = \ker\chi$. Entonces por el primer teorema de isomorfía tenemos que
    \begin{gather*}
        \frac{\bb{Z}}{p\bb{Z}} = \frac{\bb{Z}}{\ker\chi} \cong  Im\chi \leq K
    \end{gather*}
    Donde la última inclusión es de subanillo. Como $Im\chi$ es un dominio de integridad tendremos que $p=0$ o, si $p>0$, entonces $p$ es primo.
\end{observacion}

\begin{prop}
    Sea $K$ un cuerpo de característica $p$, entonces, 
    \begin{itemize}
        \item si $p>0$, el subcuerpo primo de $K$ es isomorfo a $\bb{Z}_p$
        \item si $p=0$, el subcuerpo primo de $K$ es isomorfo a $\bb{Q}$
    \end{itemize}
    \begin{proof}
        Denotamos por $\Pi$ al subcuerpo primo de $K$.
        \begin{itemize}
            \item Si $p>0$, entonces $Im\chi$ es un subcuerpo de $K$ $\Rightarrow \Pi\subseteq Im\chi$, pero $Im\chi\cong \bb{Z}_p$ y como $\bb{Z}_p$ no tiene subcuerpos propios, entonces $\Pi = Im\chi \cong\bb{Z}_p$
            
            \item Si $p=0$, entonces $\bb{Z}\cong Im\chi \leq K$ (subanillo) y entonces $Im\chi\subseteq \Pi$, ya que $Im\chi$ es el subanillo más pequeño. Si $Q$ es el cuerpo de funciones de $Im\chi$, entonces $Q\cong \bb{Q}$. Aplicando la propiedad universal del cuerpo de fracciones tenemos que $\bb{Q}\subseteq \Pi$ por lo que $\bb{Q}=\Pi$ por unicidad del cuerpo de fracciones excepto isomorfismos.
        \end{itemize}
    \end{proof}
\end{prop}

\begin{definicion}[Extensión de cuerpos]
    Sea $F$ un subcuerpo de $K$, diremos que $F\leq K$ es una \textbf{extensión de cuerpos}.
\end{definicion}

\begin{observacion}
    Sea $F\leq K$ una extensión, entonces $K$ es un espacio vectorial sobre $F$ donde 
    \begin{itemize}
        \item  la suma de $K$ es la suma como espacio vectorial
        \item la acción de los escalares, $\lambda \in F$, $\alpha \in K$,  $\lambda\alpha$ es el producto en $K$
    \end{itemize} 
\end{observacion}

\begin{definicion}
    Sea $\bb{R}\leq K$ una extensión, entonces la dimensión de $K$ sobre $F$ (como espacio vectorial) se llama \textbf{grado} de la extensión $F\leq K$ y se denota por $[K:F]$, es decir
    \begin{gather*}
        [K:F] = \dim_F(K)
    \end{gather*}
\end{definicion}

\begin{ejemplo}\
    \begin{itemize}
        \item $[\bb{C}: \bb{R}] = 2$
        \item $[\bb{R}: \bb{Q}] = \infty$, ya que $\bb{R}$ no es numerable
    \end{itemize}
\end{ejemplo}

\begin{notacion}
    Si $[K:F]<\infty$ diremos que $F\leq K$ es finita. Si $[K:F]=\infty$ diremos que $F\leq K$ no es finita o es infinita.
\end{notacion}

\begin{ejercicio}
    Demostrar que el cardinal de un cuerpo finito es de la forma $p^n$ con $p$ primo y $n\geq 1$.

    % el Subcuerpo primo de un cuerpo finito tiene característica p primo por lo que mi cuerpo abstracto finito tiene que ser un espacio vectorial sobre Z_p de dimensión finita n. Sabemos que es isomorfo como espacio vectorial a Z_p x Z_p x ... x Z_p n veces por lo que el cardinal es p^n.
\end{ejercicio}

\begin{notacion}
    Sea la extensión $F\subseteq K$ y $S\subseteq K$ un subconjunto de $K$. Podemos considerar el menor subcuerpo de $K$ que contiene a $F\cup S$ y lo denotaremos por $F(S)$ y lo llamaremos \textbf{extensión de $F$ generada por $S$} (dentro de $K$). Si $S$ es finito, es decir, $S=\{s_1,\dots,s_t\}$ simplifico la notación como $F(\{s_1,\dots,s_t\}) = F(s_1,\dots,s_t)$
\end{notacion}

\begin{ejemplo}
    $\bb{Q}(\sqrt(2))$ donde $\sqrt{2}\in \bb{R}$, es decir, es el menor subcuerpo de los reales que contiene a $\sqrt{2}$. Por tanto $\bb{Q}(\sqrt(2)) = \{a+b\sqrt{2} : a,b\in \bb{Q}\}$. Esto se ve fácilmente viendo la doble inclusión. La inclución $\supseteq$ es obvia y demostrando que $\{a+b\sqrt{2} : a,b\in \bb{Q}\}$ es un subcuerpo tenemos automáticamente la igualdad. Esta extensión tendrá grado 2.
\end{ejemplo}

\begin{definicion}
    Sea $K$ un cuerpo, consideramos el cuerpo de polinomios con coeficientes en $K$, y lo denotamos por $K[x]$. \\
    
    Dado un $f\in K[x]$ y $K\leq E$ una extensión de cuerpos tal que $f$ se descompone completamente en $E[X]$ como producto de polinomios lineales\footnote{de grado 1} y $E=K(\alpha_1,\dots,\alpha_t)$ con $\alpha_1, \dots, \alpha_t\in E$ las raíces de $f$, entonces diremos que $E$ es un \textbf{cuerpo de descomposición} (de escisión) de $f$ (sobre $K$).
\end{definicion}

\begin{ejemplo}
    Consideramos el polinomio $x^2 + 1 \in \bb{R}[x]$ que es irreducible\footnote{no tiene raíces en $\bb{R}$ y no se puede descomponer en producto de polinomios de grado menor} sobre $\bb{R}$. Un cuerpo de descomposición suyo es $\bb{C}$.\\

    Podemos considerar además $x^2+1 \in \bb{Q}[x]$ y entonces el c.d.d\footnote{cuerpo de descomposición} es $\bb{Q}(i)$ (y además $[\bb{C} : \bb{Q}(i)]=\infty$ y se deja esto como ejercicio).
    \begin{gather*}
        x^2 +1 = (x-i)(x+i)
    \end{gather*}
\end{ejemplo}

\begin{observacion}
    Si $f\in Q[x]$, entonces tomo todas sus raíces en $\bb{C}$, digamos $\alpha_1, \dots, \alpha_t$ y c.d.d de $f$ es $Q(\alpha_1, \dots, \alpha_t)$
\end{observacion}


