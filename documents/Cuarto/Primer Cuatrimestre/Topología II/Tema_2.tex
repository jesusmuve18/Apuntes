\chapter{Espacios Recubridores}

\begin{observacion}
    A lo largo de este tema supondremos que todos los espacios topológicos son conexos y localmente arcoconexos. En particular estos espacios son siempre arcoconexos.
\end{observacion}

\section{Levantamiento de aplicaciones}

\begin{observacion}
    Vamos a tener en cuenta que si $X$ es un espacio topológico conexo y localmente arcoconexo, entonces todo abierto suyo cumple que cada componente arcoconexa es abierta.

    \begin{proof}
        Si $O$ es abierto y $A$ es una componente arcoconexa de $O$, entonces dado $a\in A$, como $X$ es localmente arcoconexo tendremos que existe un $U$ entorno arcoconexo de $a$ tal que $U\subseteq O$. Como $A$ es el mayor arcoconexo en $O$ que contiene al punto $a$ tendremos que $U\subseteq A$, luego $A$ es abierto.
    \end{proof}
    
    Esto lo vamos a usar para el caso en el que tenemos dada una aplicación recubridora $p:R\to B$ y un punto $b_0\in B$. Entonces tendremos que existe un abierto regularmente recubierto $O$ que contiene a $b_0$. Restringiéndonos a la componente arcoconexa de $O$ que contiene a $b_0$ podremos suponer que el entorno regularmente recubierto es abierto y arcoconexo.
\end{observacion}

\begin{lema}[Unicidad del levantamiento]
    Sean $p:R \to B$ una aplicación recubridora, $f_1,f_2:X\to R$ continuas tales que 
    \begin{gather*}
        p\circ f_1 = p\circ f_2
    \end{gather*}
    Si existe un $x_0\in X$ tal que $f_1(x_0)=f_2(x_0)$, entonces $f_1=f_2$.

    \begin{proof}
        Para la demostración solo se necesita que $X$ sea conexo y no necesariamente localmente arcoconexo. \\

        Partimos la siguiente situación
        \begin{gather*}
            \xymatrix{
                && R \ar[d]^{p}\\
                X \ar[urr]^{f_1,f_2} \ar[rr]_{f=p\circ f_1 = p\circ f_2} && B
            }
        \end{gather*}
        Consideramos el siguiente conjunto
        \begin{gather*}
            Y=\{x\in X : f_1(x)=f_2(X)\}
        \end{gather*}
        Como $X$ es conexo y tenemos que $Y \neq \emptyset$, ya que por hipótesis $x_0\in Y$, si probamos que $Y$ es abierto y cerrado tendremos que $Y=X$, es decir, $f_1=f_2$. \\
        
        Veamos que $Y$ es abierto. Para ello tomamos $y\in Y$, es decir, un punto $y$ tal que $f_1(y)=f_2(y)$. Elegimos el punto $b=p(f_1(y))=p(f_2(y))$. Sea $O$ abierto regularmente recubierto y arcoconexo que contiene a $b$, entonces 
        \begin{gather*}
            p^{-1}(O) = \bigcup\limits_{i\in I} A_i
        \end{gather*}
        donde los $A_i$ son abiertos disjuntos de $R$ y tal que 
        \begin{gather*}
            p_{|A_i}: A_i \to O
        \end{gather*} 
        es un homeomorfismo. Tomamos el abierto $A_{i_0}$ donde se encuentra $f_1(y)=f_2(y)$. Elegimos $V=f_1^{-1}(A_{i_0}) \cap f_2^{-1}(A_{i_0})$. Veamos que $\forall x \in V$ se tiene que $f_1(x)=f_2(x)$. Como $x\in V$ tendremos que $f_1(x),f_2(x)\in A_{i_0}$ por lo que 
        \begin{gather*}
            p(f_1(x)) = p(f_2(x)) \overset{(\ast)}{\Rightarrow} f_1(x)=f_2(x) \Rightarrow V\subseteq Y
        \end{gather*}
        donde en $(\ast)$ hemos usado que $p_{|A_i}$ es inyectiva. Tenemos finalmente que $Y$ es abierto.\\

        Veamos ahora que $Y$ es cerrado. Para ello demostramos que $X\setminus Y$ es abierto. Tomamos $y\in X\setminus Y$ y vemos que existe un $V$ abierto que contiene al punto $y$ y tal que $V\subseteq X \setminus Y$. Sea $b=p(f_1(y))=p(f_2(y))$ y de nuevo tomamos $O$ regularmente recubierto que contiene a $b$. Tendremos
        \begin{gather*}
            p^{-1}(O) = \bigcap\limits_{i\in I} A_i
        \end{gather*}
        donde los $A_i$ son abiertos disjuntos y tal que 
        \begin{gather*}
            p_{|A_i}: A_i\to O 
        \end{gather*}
        es un homeomorfismo. Tendremos $f_1(y)\in A_{i_1}$ y $f_2\in A_{i_2}$ y además se verificará que $A_{i_1}\neq A_{i_2}$ ya que si se diera la igualdad tendríamos que la aplicación
        \begin{gather*}
            p_{|A_{i_1}}: A_{i_1} \to O
        \end{gather*}
        no sería intectiva. Elegimos ahora $V=f_1^{-1}(A_1) \cap f_2^{-1}(A_2)$, donde se tiene que $y\in V$. Además se tiene que 
        \begin{gather*}
            f_1(V)\subseteq A_{i_1}\\
            f_2(V)\subseteq A_{i_2}
        \end{gather*}
        por lo que para cada $x\in V$ se tendrá que $f_1(x)\neq f_2(x)$ ya que $f_1(x)\in A_{i_1}$ y $f_2(x)\in A_{i_2}$. Esto nos dice que $V\subseteq X\setminus Y$, luego $Y$ es cerrado.
    \end{proof}
\end{lema}

\begin{teo}[Teorema de monodromía]
    Sean $p:R \to B$ una aplicación recubridora, $b_0\in V$ y $r_0\in p^{-1}(b_0)$. El homomorfismo inducido $p_*:\pi_1(R,r_0)\to \pi_1(B,b_0)$ es inyectivo. En partircular, $\pi_1(R,r_0)$ es isomorfo a $p_*(\pi_1(R,r_0))< \pi_1(B,b_0)$.

    \begin{proof}
        Sabemos que $p_*$ es inyectiva si y solo si $\ker(p_*)$ es trivial. Tomamos $\alpha$ lazo basado en $r_0$ tal que 
        \begin{gather*}
            p_*([\alpha]) = [\veps_{b_0}]
        \end{gather*}
        Como además $[p\circ \alpha] = p^*([\alpha])$ tenemos que existe una homotopía por lazos de $\veps_{b_0}$ en $p\circ \alpha$. Como toda homotopía por arcos se puede levantar tenemos que existe una homotopía por arcos en $R$ de $\hat{\veps}_{b_0}$ y $\widehat{p\circ \alpha}$ (empezando en $r_0$). Tenemos
        \begin{gather*}
            \left.
            \begin{array}{l}
                \hat{\veps}_{b_0} = \veps_{r_0}\\\\
                \widehat{p\circ \alpha} = \alpha
            \end{array}
            \right\}  \Rightarrow [\alpha] = [\veps_{r_0}]
        \end{gather*}
    \end{proof}
\end{teo}

\begin{observacion}
    Recordemos que dados dos subgrupos $H_1,H_2$ de un grupo $G$ se dice que $H_1$ y $H_2$ son conjugados si existe un $g\in G$ tal que 
    \begin{gather*}
        H_2=g^{-1}H_1 g
    \end{gather*}
\end{observacion}

\begin{coro}
    Sean $p:R \to B$ una aplicación recubridora, $b_0\in B$ y $r_1,r_2\in p^{-1}(b_0)$. Elegimos un arco $\alpha:[0,1]\to R$ tal que
    \begin{gather*}
        \alpha(0)=r_1\\
        \alpha(1)=r_2
    \end{gather*}
    entonces 
    \begin{gather*}
        p_*(\pi_1(R,r_2)) = [p\circ\alpha]^{-1} \ast p_*(\pi_1(R,r_1)) \ast [p\circ\alpha]
    \end{gather*}
    En particular, $p_*(\pi_1(R,r_1))$ y $p_*(\pi_1(R,r_2))$ son conjugados en $\pi_1(B,b_0)$.

    \begin{proof}
        Sabemos que $p\circ \alpha$ es un lazo basado en $b_0$ por lo que $[p\circ \alpha]\in \pi_1(B,b_0)$. Además,
        \begin{align*}
            \pi_1(R,r_2) &\overset{isom.}{\to} \pi_1(R,r_1)\\
            [\beta] &\mapsto [\alpha \ast \beta \ast \tilde{\alpha}]
        \end{align*}
        Tenemos por tanto que 
        \begin{gather*}
            \pi_1(R,r_1) = [\alpha] \ast \pi_1(R,r_2) \ast [\tilde{\alpha}]\\
            p_*(\pi_1(R,r_1)) = [p\circ \alpha] \ast p_*(\pi_1(R,r_2)) \ast [p\circ \tilde{\alpha}]
        \end{gather*}
        Como tenemos que 
        \begin{gather*}
            [p\circ \tilde{\alpha}] = [\tilde{p\circ \alpha}] = [p\circ \alpha]^{-1}
        \end{gather*}
        llegamos a que son conjugados.
    \end{proof}
\end{coro}

\begin{coro}
    Sean $p:R\to B$ una aplicación recubridora, $b_0\in B$ y $r_1\in p^{-1}(b_0)$. Sea $H$ un subgrupo conjugado de $p_*(\pi_1(R,r_1))$ en $\pi_1(B,b_0)$. Entonces existe un punto $r_2\in R$ tal que
    \begin{gather*}
        H=p_*(\pi_1(R,r_2))
    \end{gather*}
    \begin{proof}
        Por hipótesis sabemos que $p(r_1)=b_0$ y que $p_*(\pi_1(R,r_1))$ es conjugado con $H$ en $\pi_1(B,b_0)$, es decir, 
        \begin{gather*}
            H = g^{-1} \ast p_*(\pi_1(R,r_1)) \ast g
        \end{gather*}
        con $g\in \pi_1(B,b_0)$, esto es, $g=[\gamma]$. Consideramos $\hat{\gamma}$ el levantamiento de $\gamma$ a $R$ con 
        \begin{gather*}
            \hat{\gamma}(0) = r_1
        \end{gather*}
        y llamamos $r_2=\hat{\gamma}(1)$ al final del arco.
        \begin{gather*}
            p(r_2) = (p\circ \hat{\gamma})(1) = \gamma(1) = b_0
        \end{gather*}
        Usando el corolario anterior tenemos que 
        \begin{align*}
            p_*(\pi_1(R,r_2)) &= [p\circ \hat{\gamma}]^{-1} \ast p_*(\pi_1(R,r_1)) \ast [p\circ \hat{\gamma}] =\\
            &= [\gamma]^{-1} \ast p_*(\pi_1(R,r_1)) \ast [\gamma] =\\
            &= H
        \end{align*}
    \end{proof}
\end{coro}

\begin{teo}
    Consideramos una aplicación recubridora $p:R\to B$, una aplicación continua $f:X\to B$, $x_0\in X$, $b_0=f(x_0)$ y $r_0\in p^{-1}(b_0)$. 
    
    \begin{gather*}
        \xymatrix{
            & R \ar[d]^p\\
            X \ar[r]^{f} \ar@{-->}[ur]^{\hat{f}} & B
        }
    \end{gather*}

    Entonces son equivalentes:
    \begin{enumerate}
        \item Existe un levantamiento $\hat{f}:X \to R$ de $f$ con $\hat{f}(x_0)=r_0$.
        \item $f_*(\pi_1(X,x_0))\subset p_*(\pi_1(R,r_0))$
    \end{enumerate}
    Además, si se cumple cualquiera de estas condiciones, el levantamiento $\hat{f}$ de $f$ con $\hat{f}(x_0)=r_0$ es único.
\end{teo}

\begin{observacion}
    Una consecuencia inmediata es que si $X$ es simplemente conexo, toda $X:X\to B$ continua se puede levantar.
\end{observacion}