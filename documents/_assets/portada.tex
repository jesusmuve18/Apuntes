\usepackage{xkeyval} % Para el paso de argumentos

\usepackage{graphicx}

% Definir la carpeta de las imágenes
\graphicspath{{../_assets}{../../_assets}{../../../_assets}{../../../../_assets}{../../../../../_assets}}

% Definir el comando \portada
\makeatletter
\define@key{portada}{titulo}{\def\titulo{#1}}
\define@key{portada}{subtitulo}{\def\subtitulo{#1}}
\define@key{portada}{autor}{\def\autor{#1}}
\define@key{portada}{año}{\def\año{#1}}
\define@key{portada}{color página}{\def\colorpagina{#1}}
\define@key{portada}{imagen}{\def\imagen{#1}}

\newcommand*{\portada}[1][]{%
  % Encabezado
  \fancyhead[L]{\helv \titulo}
  \fancyhead[R]{\helv \nouppercase{\leftmark}}

    % Definimos las claves y sus valores por defecto
  \setkeys{portada}{%
    titulo=Sin Título,%
    subtitulo=Sin Subtítulo,%
    autor=Autor Desconocido,%
    año=Sin Año,%
    color página=white,%
    imagen=Logo-UGR-Black.png, #1}%
    \begin{titlepage}
        \centering
        \pagecolor{\colorpagina} % Cambia el color de fondo de la portada
        {\includegraphics[width=0.2\textwidth]{\imagen}\par}
        \vspace{1cm}
        {\bfseries\LARGE Universidad de Granada \par}
        \vspace{1cm}
        {\scshape\Large Doble Grado en Ingeniería Informática y Matemáticas \par}
        \vspace{3cm}
        {\scshape\Huge \titulo \par}
        \vspace{3cm}
        {\itshape\Large \subtitulo \par}
        \vfill
        {\Large Autor: \par}
        {\Large \autor \par}
        \vfill
        {\Large \año \par}
    \end{titlepage}%
    \pagecolor{white}
}